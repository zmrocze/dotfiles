%!TEX root = main.tex

% Logics
% Description logics:
\newcommand{\DL}[1]{\ensuremath{\mathcal{#1}}}  % Description Logics

% The ALC family of DLs
\newcommand{\ALC}{\DL{ALC}}                     % DL ALC
\newcommand{\ALCI}{\DL{ALCI}}                   % DL ALCI
\newcommand{\ALCO}{\DL{ALCO}}                   % DL ALCO
\newcommand{\ALCHQ}{\DL{ALCHQ}}                 % DL ALCHQ
\newcommand{\Self}{\mathsf{Self}}
\newcommand{\ALCHbself}{\DL{ALCH}\textit{b}_{\Self}}  % DL ALCIHbself
\newcommand{\ALCSelf}{\DL{ALC}_{\Self}}  % DL ALCIHbself
\newcommand{\ZOI}{\DL{ZOI}} 
\newcommand{\Z}{\DL{Z}} 

\newcommand{\ALCHbselfreg}{\DL{ALCH}\textit{b}^{\Self}_{\mathsf{reg}}}



\newcommand{\FO}{\DL{FO}}     % First-Order Logic
\newcommand{\GF}{\DL{GF}}     % Guarded Fragment
\newcommand{\FF}{\DL{FF}}     % Forward Fragment
\newcommand{\FL}{\DL{FL}}     % Fluted Fragment
\newcommand{\OF}{\DL{OF}}     % Ordered Fragment
\newcommand{\FGF}{\DL{FGF}}   % Forward Guarded Fragment
\newcommand{\UNFO}{\DL{UNFO}} % Unary Negation Fragment

% Complexity classses:
\newcommand{\complexityclass}[1]{\textsc{#1}} % any complexity class
\newcommand{\NP}{\complexityclass{NP}}
\newcommand{\coNP}{\text{co}\complexityclass{NP}}
\newcommand{\PTime}{\complexityclass{PTime}}
\newcommand{\PSpace}{\complexityclass{PSpace}}
\newcommand{\APSpace}{\complexityclass{APSpace}}
\newcommand{\ExpTime}{\complexityclass{ExpTime}} % exponential time
\newcommand{\ExpSpace}{\complexityclass{ExpSpace}} % exponential space
\newcommand{\NExpTime}{\complexityclass{NExpTime}} % nondeterministic exponential time
\hyphenation{Exp-Time} % prevent "Ex-PTime" (see, e.g. Tobies'01, Glimm'07 ;-)
\hyphenation{NExp-Time} % prevent "NEx-PTime" (see, e.g. Tobies'01, Glimm'07 ;-)
\newcommand{\TwoExpTime}{\complexityclass{2ExpTime}} % doubly-exponential time
\newcommand{\Tower}{\complexityclass{Tower}} % Tower
\newcommand{\AExpSpace}{\complexityclass{AExpSpace}}

% Others
\newcommand{\str}[1]{{\mathfrak{#1}}}
\newcommand{\deff}{\; \stackrel{\text{def}}{=} \;}
\newcommand{\arity}{\mathsf{ar}}

\newcommand{\N}{{\mathbb{N}}}
\newcommand{\Q}{{\mathbb{Q}}} 
\newcommand{\Var}{\mathrm{Var}}
\newcommand{\sigSigma}{\Sigma}

% Rel symbols
\newcommand{\relsymbol}[1]{\mathrm{#1}}
\newcommand{\relsymbolP}{\relsymbol{P}}
\newcommand{\relsymbolR}{\relsymbol{R}}
\newcommand{\relsymbolS}{\relsymbol{S}}
\newcommand{\relsymbolQ}{\relsymbol{Q}}
\newcommand{\relsymbolT}{\relsymbol{T}}
\newcommand{\relsymbolU}{\relsymbol{U}}
\newcommand{\relsymbolA}{\relsymbol{A}}
\newcommand{\relsymbolB}{\relsymbol{B}}
\newcommand{\relsymbolC}{\relsymbol{C}}
\newcommand{\relsymbolD}{\relsymbol{D}}
\newcommand{\relsymbolE}{\relsymbol{E}}

% Domain elements
\newcommand{\domelem}[1]{\mathrm{#1}}                           % domain element
\newcommand{\domelemc}{\domelem{c}}                             % domain element c
\newcommand{\domelemd}{\domelem{d}}                             % domain element d
\newcommand{\domeleme}{\domelem{e}}                             % domain element e

\newcommand{\domelemtuplec}{\vec{\domelemc}}                         % tuple of domain element c
\newcommand{\domelemtupled}{\vec{\domelemd}}                         % tuple of domain element d
\newcommand{\domelemtuplee}{\vec{\domeleme}}                         % tuple of domain element e

\newcommand{\domelemtupledfromto}[2]{\vec{\domelemd}_{#1\ldots#2}}  % tuple of domelements d from #1 to #2
\newcommand{\domelemtupleefromto}[2]{\vec{\domeleme}_{#1\ldots#2}}  % tuple of domelements e from #1 to #2

% Variables:
\newcommand{\var}[1]{\mathit{#1}}       % variable
\newcommand{\varx}{\var{x}}             % variable x
\newcommand{\vary}{\var{y}}             % variable y
\newcommand{\varz}{\var{z}}             % variable z
\newcommand{\varv}{\var{v}}             % variable v
\newcommand{\varu}{\var{u}}             % variable u
\newcommand{\varw}{\var{w}}             % variable w
\newcommand{\varh}{\var{h}}             % variable h
\newcommand{\vartuplex}{\vec{\varx}}    % tuple of variables x
\newcommand{\vartuplexomega}{\vec{\varx_{\omega}}}      % tuple of variables x_omega
\newcommand{\vartupley}{\vec{\vary}}                    % tuple of variables y
\newcommand{\vartupleyone}{\vec{\vary_1}}                    % tuple of variables y_1
\newcommand{\vartupleytwo}{\vec{\vary_2}}                    % tuple of variables y_2
\newcommand{\vartuplez}{\vec{\varz}}                    % tuple of variables z
\newcommand{\vartuplev}{\vec{\varv}}                    % tuple of variables v
\newcommand{\vartupleu}{\vec{\varu}}                    % tuple of variables u
\newcommand{\vartuplew}{\vec{\varw}}                    % tuple of variables w
\newcommand{\vartupleh}{\vec{\varh}}                    % tuple of variables h

\newcommand{\vartuplexfromto}[2]{\vec{\varx}_{#1\ldots#2}}  % tuple of variables x from #1 to #2
\newcommand{\vartuplevfromto}[2]{\vec{\varv}_{#1\ldots#2}}  % tuple of variables v from #1 to #2

% Knowledge bases
\newcommand{\kb}[1]{\mathcal{#1}}   % knowledge base
\newcommand{\kbK}{\kb{K}}           % knowledge base K
\newcommand{\kbKsimplied}{\kbK_{\textit{smpl}}}           % simplified knowledge base K

% Theory
\newcommand{\theory}[1]{\mathcal{#1}}   % theory
\newcommand{\theoryT}{\theory{T}}       % theory T
\newcommand{\db}[1]{\mathcal{#1}}       % database
\newcommand{\dbD}{\db{D}}               % database D
\newcommand{\theoryC}{\mathcal{C}}   % inv-constraints C

% Queries:
\newcommand{\queryatom}{\alpha}      % query atom
\newcommand{\query}[1]{\mathit{#1}}  % query
\newcommand{\queryq}{\query{q}}      % query q
\newcommand{\match}[1]{#1}          % query match
\newcommand{\matchpi}{\match{\pi}}  % query match pi
\newcommand{\matcheta}{\match{\eta}}  % query match eta
\newcommand{\modelsmatch}[1]{\models_{#1}} % |=_match

% Languages:
\newcommand{\lang}[1]{\mathbf{#1}}  % any "language" name
\newcommand{\Ilang}{\lang{N_I}}     % the set of all individual names
\newcommand{\Rlang}{\lang{N_R}}     % the set of all relational symbols
\newcommand{\Vlang}{\lang{N_V}}     % the set of all variable names

% Individuals:
\newcommand{\indv}[1]{\mathtt{#1}}  % individual name
\newcommand{\indva}{\indv{a}}       % individual name a
\newcommand{\indvb}{\indv{b}}       % individual name b
\newcommand{\indvc}{\indv{c}}       % individual name c
\newcommand{\indvd}{\indv{d}}       % individual name d
\newcommand{\indvtuplea}{\vec{\indva}}
\newcommand{\indvtupleb}{\vec{\indvb}}
\newcommand{\indvtuplec}{\vec{\indvc}}
\newcommand{\indfromdb}{\mathsf{ind}}

% Types
\newcommand{\tp}[3]{\mathsf{tp}^{#1}_{#2}(#3)}

% Restrictions
\renewcommand{\restriction}{\mathord{\upharpoonright}}
\newcommand{\restr}[2]{#1\restriction_{#2}} % the restriction of #1 to #2

% Query graphs: 
\newcommand{\queryVar}[1]{\mathrm{Var}{(#1)}}   % Variables from a query
\newcommand{\queryVarq}{\queryVar{\queryq}}     % Variables from a query q

% Rolling-up technique

\newcommand{\clos}{\mathsf{cl}}
\newcommand{\hist}{\mathsf{hist}} % history of a tuple
\newcommand{\subtreePredicate}[2]{\relsymbol{Subt}_{#1}^{#2}}  % subtree of a query #1 rooted at var #2
\newcommand{\matchPredicate}[1]{\relsymbol{Match}_{#1}}          % match of a query #1s

% Fork rewritings

\newcommand{\eliminateforkto}{\leadsto_{\mathsf{fe}}}        % eliminate fork relation rewritting
\newcommand{\maximalforkrew}[1]{\mathsf{maxfr}{(#1)}}  % Maximal fork rewritting


% morphisms
\newcommand{\homo}[1]{\mathfrak{#1}}    % homomorphism
\newcommand{\homof}{\homo{f}}           % homomorphism f
\newcommand{\homog}{\homo{g}}           % homomorphism g
\newcommand{\homoh}{\homo{h}}           % homomorphism h
\newcommand{\ishomoto}{\vartriangleleft} % is homomorhic to
\newcommand{\homeq}{\rightleftarrows} % homomorphically equivalent
\newcommand{\isoeq}{\cong} % isomorphic

% Splittings

\newcommand{\splittingof}[1]{\Pi_{#1}}                          % Splitting of a query #1
\newcommand{\splittingofq}{\splittingof{\queryq}}                               % splitting of query q
\newcommand{\splittingtrees}{\mathrm{Trees}}                    % Splitting component: Trees
\newcommand{\splittingroots}{\mathrm{Roots}}                    % Splitting component: Roots
\newcommand{\splittingithsubtree}[1]{\mathrm{SubTree}_{#1}}     % Splitting component: #1-th Subtree
\newcommand{\splittingname}{\mathrm{name}}                      % Splitting component: naming function
\newcommand{\splittingrootof}{\mathrm{root}\text{-}\mathit{of}} % Splitting component: root-of function

% Spoilers

\newcommand{\spoil}{\text{\lightning}}
\newcommand{\superspoil}{{\spoil}^{*}}

\newcommand{\spoilerKBof}[1]{\kbK_{#1}^{\spoil}}
\newcommand{\spoilerKBofq}{\spoilerKBof{\queryq}}
\newcommand{\spoilerTheoryof}[1]{\theoryT_{#1}^{\spoil}}
\newcommand{\spoilerDBof}[1]{\dbD_{#1}^{\spoil}}
\newcommand{\superspoilerKBof}[1]{\kbK_{#1}^{\superspoil}}

