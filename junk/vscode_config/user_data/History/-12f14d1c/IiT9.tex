% Opcje klasy 'iithesis' opisane sa w komentarzach w pliku klasy. Za ich pomoca
% ustawia sie przede wszystkim jezyk oraz rodzaj (lic/inz/mgr) pracy.
\documentclass[shortabstract]{iithesis}

\usepackage[utf8]{inputenc}

%%%%% DANE DO STRONY TYTUŁOWEJ
% Niezaleznie od jezyka pracy wybranego w opcjach klasy, tytul i streszczenie
% pracy nalezy podac zarowno w jezyku polskim, jak i angielskim.
% Pamietaj o madrym (zgodnym z logicznym rozbiorem zdania oraz estetyka) recznym
% zlamaniu wierszy w temacie pracy, zwlaszcza tego w jezyku pracy. Uzyj do tego
% polecenia \fmlinebreak.
\polishtitle    {Wymagający złamania wierszy\fmlinebreak tytuł pracy w~języku polskim}
\englishtitle   {English title}
\polishabstract {\ldots}
\englishabstract{\ldots}
% w pracach wielu autorow nazwiska mozna oddzielic poleceniem \and
\author         {Maksymilian Debeściak}
% w przypadku kilku promotorow, lub koniecznosci podania ich afiliacji, linie
% w ponizszym poleceniu mozna zlamac poleceniem \fmlinebreak
\advisor        {dr Jan Kowalski}
%\date          {}                     % Data zlozenia pracy
% Dane do oswiadczenia o autorskim wykonaniu
%\transcriptnum {}                     % Numer indeksu
%\advisorgen    {dr. Jana Kowalskiego} % Nazwisko promotora w dopelniaczu
%%%%%

%%%%% WLASNE DODATKOWE PAKIETY
%
%\usepackage{graphicx,listings,amsmath,amssymb,amsthm,amsfonts,tikz}
%
%%%%% WŁASNE DEFINICJE I POLECENIA
%
%\theoremstyle{definition} \newtheorem{definition}{Definition}[chapter]
%\theoremstyle{remark} \newtheorem{remark}[definition]{Observation}
%\theoremstyle{plain} \newtheorem{theorem}[definition]{Theorem}
%\theoremstyle{plain} \newtheorem{lemma}[definition]{Lemma}
%\renewcommand \qedsymbol {\ensuremath{\square}}
% ...
%%%%%

\begin{document}

%%%%% POCZĄTEK ZASADNICZEGO TEKSTU PRACY

\chapter{Wprowadzenie}

\ldots

%%%%% BIBLIOGRAFIA
\documentclass{article}
\usepackage{amsmath}
\usepackage{amssymb}
\usepackage{amsfonts}
\usepackage{algorithm}
\usepackage[noend]{algpseudocode}
% \usepackage[utf8]{inputenc}
\usepackage[T1]{fontenc}
\usepackage{amsthm}
\newtheorem*{lemat}{Lemat}
\usepackage{mathtools}
\DeclarePairedDelimiter\ceil{\lceil}{\rceil}
\DeclarePairedDelimiter\floor{\lfloor}{\rfloor}
\newcommand{\rpm}{\raisebox{.2ex}{$\scriptstyle\pm$}}

\begin{document}



\end{document}
%\begin{thebibliography}{1}
%\bibitem{example} \ldots
%\end{thebibliography}

Let us now consider the clause $c'=(c_1\setminus\{A\})\tau\sigma \cup (c_2\setminus\{B\})\sigma$.
This is not necessarily the clause $c$ as the literals $A\tau$ and $B$ are not guaranteed to be normalized after applying $\sigma$.
Instead $c=$,  Let $\epsilon$ be the normalization of $B\sigma=B$. Then $c'\epsilon=c$, because 
We will use the Lemma~\ref{lem:normalization}. The literal $B\sigma=B$ is greater in the $\sqsubset$ order 
than every literal in $c'$, because it contains all variables of $c'$.
The lemma states that if $c'$ is forward and $\epsilon$ is the normalization of $B\sigma$.
 The substitution $\sigma$ is the unifier used in the resolution procedure, because $B$ is normalized after applying $\sigma$.
Therefore every literal in $c$ is either a literal of $c_2$ and therefore $forward$, or it is 
literal of $c_1$ after applying substitution $\sigma$.
The literal $A$ is a maximal literal in $c_1$, so $A\tau$ is a maximal literal in $c_2\tau$.
Therefore every literal in $c_1\tau$ 
\begin{itemize}
    \item does not contain non-ground function terms
    \item contains only variables from $y_{}$$A\tau$
\end{itemize}
Therefore every literal in $c_1$ is of form $(\lnot)R(\bar{y}_{i..j})$ for some infix $\bar{y}_{i..j}$ of $\bar{y}_{k..l}$.
After substitution every literal is of form $(\lnot)R(\bar{t})$ for some infix $\bar{t}$ of $\bar{x}_{k+s..o}, \bar{x}^{\bar{\alpha'}}_{{o+1..m+s}}(\bar{x}_{1..o})$, therefore is $forward$.


\end{document}
