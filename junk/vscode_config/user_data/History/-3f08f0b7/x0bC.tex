\documentclass[final]{beamerikz}
\usepackage{polski}
\usepackage{lmodern}
\usepackage{amsmath}
\usepackage[utf8]{inputenc}
\newcommand{\art}{\textit{artst}}
\newcommand{\adm}{\textit{podz}}
\newcommand{\bkpr}{\textit{pszczarz}}
\newcommand{\env}{\textit{zazdr}}
\definecolor{colone}{RGB}{205,133,63}
\newcommand{\colonecmd}{colone}
\newcommand{\hlone}[1]{\textcolor{\colonecmd}{#1}}
\definecolor{coltwo}{RGB}{100,149,237}
\newcommand{\coltwocmd}{coltwo}
\newcommand{\hltwo}[1]{\textcolor{\coltwocmd}{#1}}
\definecolor{colthree}{RGB}{34,139,34}
\newcommand{\colthreecmd}{colthree}
\newcommand{\hlthree}[1]{\textcolor{\colthreecmd}{#1}}
\definecolor{colfour}{RGB}{219,112,147}
\newcommand{\colfourcmd}{colfour}
\newcommand{\hlfour}[1]{\textcolor{\colfourcmd}{#1}}
\newcommand{\hlfive}[1]{\hl{#1}}
\newcommand{\Logic}[1]{\ensuremath{\mathsf{#1}}} % a Logic
\newcommand{\logicL}{\Logic{L}} % some logic L
\newcommand{\Linf}{\Logic{L}_{\mathsf{inf}}}   % L_infix
\newcommand{\Lsuf}{\Logic{L}_{\mathsf{suf}}} % L_suffix
\newcommand{\Lpref}{\Logic{L}_{\mathsf{pre}}} % L_prefix
\newcommand{\Ginf}{\Logic{G}_{\mathsf{inf}}}   % G_infix
\newcommand{\Gsuf}{\Logic{G}_{\mathsf{suf}}} % G_suffix
\newcommand{\Gpref}{\Logic{G}_{\mathsf{pre}}} % G_prefix
\input{macros}
% MODES: ready, final, draft, short, brief
% FONTS: small, basic, large
% OTHER: plain, wide

% packages --------------------------------------------------------------------

% \usepackage{}

% document data ---------------------------------------------------------------

\sBzTitle{Rezolucja dla Fragmentu Strzeżonego Przedniego}

\sBzAuthor{Karol Ochman-Milarski}

% macros ----------------------------------------------------------------------

% \newcommand{\w}{\omega}

% =============================================================================

\begin{document}

\begin{bzPlainFrame}%[show]
\bzOn{
	\iBzH[3]
	\bzCenter[scale=3.0] {}
	
	\iBzH[1]
	\bzCenter[bzEB, scale=1.4] {\bzTitle}

	\iBzH[0.5]
	\bzCenter{\bzAuthor}

	\bzCenter[scale=0.8] {\sc Uniwersytet Wrocławski}
}
\end{bzPlainFrame}

% =============================================================================

\begin{bzFrame}%[show]
\bzOn{
	\bzCenter[bzEB]{Kontekst}
	
}

\iBzH[3]

\bzOn{
	\bzItem{Szukamy rozstrzygalnych fragmentów logiki pierwszego rzędu}
}

\bzOn{
	\bzItem{Konkretnie: rozszerzenia logiki modalnej pozwalającego na relacje dowolnej arności}
}

\end{bzFrame}

% =============================================================================

\begin{bzFrame}%[show]
\bzOn{
	\bzCenter[bzEB]{Fragment Strzeżony [Andreka et al. 1998]}
}
\iBzH[1.5]

\bzOn{
	\bzItem{Nieograniczona liczba zmiennych i zagnieżdżenie kwantyfikatorów}
}

% \bzOn{
% 	\bzItem{Relatywizowanie kwantyfikacji przez atomy}
% }

\bzOn{
	\bzItem{Dostajemy z $\FO$ poprzez \hlone{relatywizacje kwantyfikatorów przez atomy}.}
}


\bzOn{
	\bzItem{$\exists{\vartupley} \; \hltwo{\alpha}(\vartuplex, \vartupley) {\land} \varphi(\vartuplex, \vartupley), \forall{\vartupley} \; \hltwo{\alpha}(\vartuplex, \vartupley) {\to} \varphi(\vartuplex, \vartupley)$ -- \hltwo{strażnik} pokrywa zmienne wolne z $\varphi$.  }
}

\bzOn{
	\bzLeft{\hlthree{Przykład} 1. Niektórzy artyści podziwiają tylko pszczelarzy }
	\bzCenter{$\exists{x} \; \hltwo{\art(x)} \land \forall{y} \; \left( \hltwo{\adm(\varx,\vary)} \to \bkpr(\vary) \right) $}
}

    \bzOn{
        \bzLeft{\hlthree{Przykład} 2. Każdy artysta zazdrości każdemu pszczelarzowi, którego podziwia}
        \bzCenter{$\forall{x}\; \hltwo{\art(x)} \to \forall{y}\;[\hltwo{\adm(\varx,\vary)} \to (\bkpr(y) \to \env(x, y))]$}
    }

    \bzOn{
        \bzLeft{\hl{Anty-Przykład} 3. Każdy artysta podziwia każdego pszczelarza}
        \bzCenter{$\forall{x}\;(\hltwo{\art(x)} \to \forall{y}\;(\bkpr(y) \to \adm(x, y)))$}
        \bzLine
    }
	\bzOn{
		\bzTheorem{Gradel}{Problem spełnialności dla $\GF$ jest \TwoExpTime-zupełny}
	}

\end{bzFrame}

\newcommand{\student}{\textit{stud}}
\newcommand{\admires}{\textit{podz}}
\newcommand{\prof}{\textit{prof}}
\newcommand{\intro}{\textit{przeds}}
\newcommand{\lecturer}{\textit{wykł}}

\begin{bzFrame}
	\bzOn{
			\bzCenter[bzEB]{$\FF$ i fragmenty uporządkowane [Herzig, Quine, B.]}
		}
	\iBzH[1.5]
    \bzOn{
        \bzItem{Fragmenty \hlone{uporządkowane} logiki pierwszego rzędu }
    }

    \bzOn{
        \bzItem{Zmienne w atomach występują w porządku kwantyfikacji}
    }

    \bzOn{
        \bzItem{jako sufiks/prefiks/infiks ciągu skwantyfikowanych zmiennych}
    }

    \bzOn{
        \bzLeft{\hlthree{Przykład} 1. Żaden student nie podziwia każdego profesora}
        \bzCenter{$\forall{x_1}(\student(x_1) \to \neg \forall x_2(\prof(x_2) \to \admires(x_1, x_2)))$}
    }

    \bzOn{
        \bzLeft{\hlthree{Przykład} 2. Żaden wykładowca nie przedstawia każdego profesora każdemu studentowi}
        \bzCenter{$\forall{x_1}(\lecturer(x_1) \to \neg \exists{x_2}(\prof(x_2) \land \forall{x_3}(\student(x_3) \to \intro(x_1, x_2, x_3))))$}
    }

    \bzOn{
        \bzLeft{\hl{Anty-Przykład} 1. $\forall{x_1} r(\hl{x_1, x_1})$}
    }
    \bzOn{
        \bzLeft{\hl{Anty-Przykład} 2. $\forall{x_1}\forall{x_2} r(x_1, x_2) \to s(\hl{x_2, x_1})$}
    }
    \bzOn{
        \bzLeft{\hl{Anty-Przykład} 3. $\forall{x_1}\forall{x_2}\forall{x_3} r(x_1, x_2) \land r(x_2, x_3) \to r(\hl{x_1, x_3})$}
        \bzLine
    }

    \iBzH[0.25]

\end{bzFrame}

\newcommand{\parent}{\textit{rodz}}
\newcommand{\child}{\textit{dziecko}}
\newcommand{\good}{\textit{dobre}}
\newcommand{\love}{\textit{koch}}
% \newcommand{\prof}{\textit{prof}}
% \newcommand{\intro}{\textit{przeds}}
% \newcommand{\lecturer}{\textit{wykł}}

\begin{bzFrame}
	\bzOn{
			\bzCenter[bzEB]{$\FGF$ [Bednarczyk] - Fragment Strzeżony Przedni }
		}
	\iBzH[1.5]
    \bzOn{
        \bzItem{Przecięcie $\GF$ i $\FF$}
    }

	\bzOn{
		\bzLeft{\hlthree{Przykład} 1. Wszyscy rodzice mają nie grzeczne dziecko}
        \bzCenter{$\forall{x_1, x_2}(\parent(x_1, x_2) \to \exists{x_3} (\child(x_1,x_2,x_3) \land \neg \good(x_3))$}
    }

	\bzOn{
		\bzLeft{\hlthree{Przykład} 1. Wszystkie dzieci mają rodziców}
        \bzCenter{$\forall{x_1, x_2, x_3}(\child(x_1, x_2, x_3) \to \parent(x_1, x_2))$}
    }


	% \bzOn{
	% 	\bzLeft{\hlthree{Przykład} 1. Niektórzy rodzice mają wnuka}
    %     \bzCenter{$\exists{x_1, x_2}(\parent(x_1, x_2) \land \exists{x_3} (\child(x_1,x_2,x_3) \land \neg \good(x_3))$}
    % }

	\bzOn{
		\bzLeft{\hl{Anty-Przykład} 1. Wszyscy rodzice kochają swoje dzieci}
        \bzCenter{$\forall{x_1, x_2}(\parent(x_1, x_2) \to \forall{x_3}(\child(x_1,x_2,x_3) \land \love(\hl{x_1,x_3}) \land \love(x_2,x_3))$}
    }

	\bzOn{
        \bzItem{Dobre własności teorio modelowe}
    }

	\bzOn{
		\bzTheorem{Bednarczyk}{Problem spełnialności dla $\FGF$ jest w \ExpTime}
	}

	\bzOn{
		\bzItem{Mało praktyczny algorytm}
	}


\end{bzFrame}

\begin{bzFrame}
	\bzOn{
			\bzCenter[bzEB]{Motywacja}
		}
	\iBzH[1.5]

	\bzOn{
		\bzItem{$\GF$ można rozstrzygać rezolucyjnie [Nivelle]}
	}

	\bzOn{
		\bzItem{Jak z $\FGF$?}
	}

\end{bzFrame}


\begin{bzFrame}
	\bzOn{
			\bzCenter[bzEB]{Nasz wkład}
		}
	\iBzH[1.5]

	\bzOn{
		\bzItem{Adaptujemy rezolucyjną procedurę dla $\GF$ do wykorzystania dla $\FGF$}
	}

	\bzOn{
		\bzItem{... przy czym algorytm nie potrzebuje żadnej adaptacji}
	}

	\bzOn{
		\bzItem{Pokazujemy, że rezolucyjna procedura dla $\GF$ nakarmiona formułą $\FGF$ zadziała w czasie pojedynczo wykładniczym}
	}

\end{bzFrame}

\newcommand{\Rrel}{\textit{R}}
\newcommand{\Qrel}{\textit{Q}}
\newcommand{\Srel}{\textit{S}}
\newcommand{\Prel}{\textit{P}}
% \newcommand{\good}{\textit{dobre}}
% \newcommand{\love}{\textit{koch}}

\begin{bzFrame}
	\bzOn{
			\bzCenter[bzEB]{Jak działa procedura?}
		}
	\iBzH[1.5]

    % \bzOn{
    %     \bzLeft{\hlthree{Przykład}}
    %     \bzCenter{$\forall{x_1,x_2}(\Rrel(x_1, x_2) \to \neg \forall{x_3,x_4} (\Qrel(x_1,x_2,x_3,x_4) \land (\Rrel(x_2,x_3) \lor \Rrel(x_3,x_4))))$}
    % }

    % \bzOn{
    %     \bzLeft{\hlthree{w postaci NNF}}
    %     \bzCenter{$\forall{x_1,x_2}(\Rrel(x_1, x_2) \to \exists{x_3,x_4} (\neg\Qrel(x_1,x_2,x_3,x_4) \lor (\neg \Rrel(x_2,x_3) \land \neg\Rrel(x_3,x_4))))$}
    % }

	% \bzOn{
    %     \bzLeft{\hlthree{dalej}}
    %     \bzCenter{$\{\forall{x_1,x_2}(\Rrel(x_1, x_2) \to \exists{x_3,x_4} (\neg\Qrel(x_1,x_2,x_3,x_4) \lor (\neg \Rrel(x_2,x_3) \land \neg\Rrel(x_3,x_4))))\}$}
    % }

	\bzOn{
        \bzLeft{\hlthree{Przykład}}
        \bzCenter{$\forall{x_1}(\Prel(x_1) \to \neg \forall{x_2} (\Rrel(x_1,x_2) \to (\exists{x_3} \Srel(x_1,x_2,x_3) \land \Qrel(x_1,x_2,x_3))))$}
    }

    \bzOn{
        \bzLeft{\hlthree{w postaci NNF}}
        \bzCenter{$\forall{x_1}(\Prel(x_1) \to \exists{x_2} (\Rrel(x_1,x_2) \land (\forall{x_3} \Srel(x_1,x_2,x_3) \to \neg\Qrel(x_1,x_2,x_3))))$}
    }


\end{bzFrame}

\begin{bzFrame}
	\bzOn{
			\bzCenter[bzEB]{Jak działa procedura?}
		}
	\iBzH[1.5]

	\bzOn{
        \bzLeft{\hlthree{Przykład}}
        \bzCenter{$\forall{x_1}(\Prel(x_1) \to \neg \forall{x_2} (\Rrel(x_1,x_2) \to (\exists{x_3} \Srel(x_1,x_2,x_3) \land \Qrel(x_1,x_2,x_3))))$}
    }

    \bzOn{
        \bzLeft{\hlthree{w postaci NNF}}
        \bzCenter{$\forall{x_1}(\Prel(x_1) \to \exists{x_2} (\Rrel(x_1,x_2) \land (\hltwo{\forall{x_3} \Srel(x_1,x_2,x_3) \to \neg\Qrel(x_1,x_2,x_3)})))$}
    }

	\bzOn{
        \bzLeft{\hlthree{dalej}}
        \bzCenter{$\bzEq{
			\forall{x_1}(\Prel(x_1) \to \exists{x_2} (\Rrel(x_1,x_2) \land (\hltwo{\mathit{A}(x_1,x_2)}))) \\
			\land \forall{x_1,x_2, x_3} \mathit{A}(x_1,x_2) \to \hltwo{\Srel(x_1,x_2,x_3) \to \neg\Qrel(x_1,x_2,x_3)}
		}$}
    }
\end{bzFrame}

\begin{bzFrame}
	\bzOn{
			\bzCenter[bzEB]{Jak działa procedura?}
		}
	\iBzH[1.5]

	\bzOn{
        \bzLeft{\hlthree{Przykład}}
        \bzCenter{$\forall{x_1}(\Prel(x_1) \to \neg \forall{x_2} (\Rrel(x_1,x_2) \to (\exists{x_3} \Srel(x_1,x_2,x_3) \land \Qrel(x_1,x_2,x_3))))$}
    }

    \bzOn{
        \bzLeft{\hlthree{w postaci NNF}}
        \bzCenter{$\forall{x_1}(\Prel(x_1) \to \exists{x_2} (\Rrel(x_1,x_2) \land (\forall{x_3} \Srel(x_1,x_2,x_3) \to \neg\Qrel(x_1,x_2,x_3))))$}
    }

	\bzOn{
        \bzLeft{\hlthree{dalej}}
        \bzCenter{$\bzEq{
			\forall{x_1}(\Prel(x_1) \to \hltwo{\exists{x_2}} (\Rrel(x_1,\hltwo{x_2}) \land (\mathit{A}(x_1,\hltwo{x_2})))) \\
			\land \forall{x_1,x_2, x_3} (\mathit{A}(x_1,x_2) \to (\Srel(x_1,x_2,x_3) \to \neg\Qrel(x_1,x_2,x_3))) \\
		}$}
    }
	\iBzH[1]
	\bzOn{
        \bzLeft{\hlthree{skolemizacja:}}
        \bzCenter{$\bzEq{
			\forall{x_1}(\Prel(x_1) \to (\Rrel(x_1,\hltwo{x_2(x_1)}) \land (\mathit{A}(x_1,\hltwo{x_2(x_1)})))) \\
			\land \forall{x_1,x_2, x_3} (\mathit{A}(x_1,x_2) \to (\Srel(x_1,x_2,x_3) \to \neg\Qrel(x_1,x_2,x_3)))
		}$}
    }
	\bzOn{
		\bzItem{
			Na końcu zamieniamy na postać koniunkji klauzul
		}
	}
\end{bzFrame}

\begin{bzFrame}
	\bzOn{
			\bzCenter[bzEB]{Rezolucja}
		}
	\iBzH[1.5]

	\bzOn{
		\bzItem{Zaczynamy od zbioru klazul otrzymanego z wyjściowej formuły}
	}

	\bzOn{
		\bzItem{Wyprowadzamy nowe, wynikające klauzule}
	}
	\bzOn{
		\bzResolution{}{Dwie klauzule: $c_1=\{A_1\} \cup R_1$ and $c_2=\{\lnot A_2\} \cup R_1$.
		Rezolwentem byłaby klauzula $R_1\theta \cup R_2\theta$.}
	}
	\iBzH[1]
	\bzOn{
		\bzFactoring{}{Klauzula: $c_1=\{B_1, B_2\} \cup R$.
		Dzielnikiem byłaby $\{A_1\theta\}\cup R\theta$.}
	}
	\bzOn{
		\bzItem{Rezolucja ograniczona porządkiem}
	}
	\bzOn{
		\bzTheorem{Nivelle}{Jeśli, wyjściowa formuła jest niespełnialna, to w wynikowym zbiorze klauzul znajdzie się klauzula pusta.}
	}
\end{bzFrame}

\begin{bzFrame}
	\bzOn{
			\bzCenter[bzEB]{Niezmiennik}
		}
	\iBzH[1.5]
	\bzOn{
        \bzLeft{\hlthree{Przykład}}
        \bzCenter{$R(x_1,x_2)$, $\neg R(x_2, x_3(x_1, x_2))$}
	}
	\bzOn{
		\bzItem{
			Termy pojawiające się w przebiegu rezolucji mają bardzo prostą postać
		}
	}
	\bzOn{
		\bzItem{
			$R(x_i, \dots, x_j, x_{j+1}(x_1, \dots, x_j), \dots, x_{k}(x_1, \dots, x_j))$
		}
	}

	\bzOn{
		\bzItem{
			Takich termów jest wielomianowo wiele!
		}
	}

\end{bzFrame}

\begin{bzFrame}
	\bzOn{
			\bzCenter[bzEB]{Implementacja}
		}
	\iBzH[1.5]

	\bzOn{
		\bzItem{Działa dla wszystlich formuł $\GF$}
	}

	\bzOn{
        \bzLeft{\hlthree{Czy jest spełnialna?}}
        \bzCenter{$\bzEq{\forall{x_1,x_2} (S(x_1,x_2) \to (P(x_1) \land \neg P(x_2))) \\
		\land \exists{x_1,x_2,x_3} (R(x_1,x_2,x_3) \land S(x_1,x_2) \land S(x_2,x_3))}$}
	}
	\iBzH[1.5]
	\bzOn{
		\bzItem{Nie!}
	}

\end{bzFrame}


\begin{bzFrame}
	\bzOn{
			\bzCenter[bzEB]{Implementacja}
		}
	\iBzH[4]

	\bzOn{
        \bzLeft{\hltwo{Etapy klauzyfikacji}}
		\node[bzG] at (0, -5) {\includegraphics[width=6cm]{scr_impl_1.png}};
	}

\end{bzFrame}

\begin{bzFrame}
	\bzOn{
			\bzCenter[bzEB]{Implementacja}
		}
	\iBzH[4]

	\bzOn{
        \bzLeft{\hltwo{Rezolucja}}
		\bzCenter{\includegraphics[scale=0.5]{scr_impl_2.png}}
	}

\end{bzFrame}

% =============================================================================

\end{document}
