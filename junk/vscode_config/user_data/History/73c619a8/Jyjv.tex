\documentclass[]{beamerikz}

\usepackage{verbatimbox}
\usepackage{fancyvrb}

\sBzTitle{\bf \beamerikz class manual}

\sBzAuthor{Micha{\l} Skrzypczak}

\newcommand{\bzCode}[1]{
	\node[bzL, anchor=north west, scale=11.03/\bzUseScale] at (\bzI+0.5, \bzH+0.18) {\linespread{1}\fontsize{10}{12}\selectfont #1};
}

\newcommand{\hint}{\hb{Hint}: }

\newcommand{\TikZ}{Ti{\it k}Z\xspace}

\begin{document}

% =======================================================================================

\begin{bzPlainFrame}%[show]
\bzOn{
	\iBzH[4.5]
	\bzCenter[bzEB, scale=1.4]{\bzTitle}

	\iBzH[1.0]
	\bzCenter[scale=1.1]{\bzAuthor}

	\bzCenter[scale=0.8] {\sc University of Warsaw}
}
\end{bzPlainFrame}

% =======================================================================================

\begin{bzFrame}%[show]
\bzOn{
	\bzCenter[bzEB]{Setup}
}

\iBzH

\bzOn{
	\bzList{Download the \Verb"beamerikz.cls" file.}
}

\bzOn{
	\bzList{Place it in the folder where your \Verb"*.tex" file is.}
}

\bzOn{
	\bzList{Prepare your slides in the \Verb"*.tex" file.}
}

\bzOn{
	\bzList{Compile them with \Verb"pdflatex" (optimally twice).}
}

\bzOn{
	\bzList{See the result.}
}

\bzOn{
	\bzList{Be happy$^\ast$.}
	\bzEvalInt{\itemHappy}{\bzL}
}

\iBzH[6]

\bzOn{
	\bzText[scale=0.7] {{\bf $^\ast$} Item {\bf \itemHappy} is optional.}
}
\end{bzFrame}

% =======================================================================================

\begin{myverbbox}{\vbMin}
\documentclass{beamerikz}              % defines the document to be BeamerikZ

\begin{document}
\begin{bzFrame}                        % the only frame of that document
\bzOn{                                 % content to appear first
    \bzCenter{Welcome to \beamerikz!}  % a centred text
}
\end{bzFrame}
\end{document}
\end{myverbbox}

\begin{myverbbox}{\vbMid}
\documentclass{beamerikz}              % defines the document to be BeamerikZ

\begin{document}
\begin{bzFrame}                        % the only frame of that document
\bzOn{                                 % content to appear first
    \bzCenter{Welcome to \beamerikz!}  % a centred text
}
\bzOn{                                 % content to appear second
    \bzCenter{Welcome again, nice to see you!}
    \bzCenter{Hope that you'll stay longer :)}
}
\end{bzFrame}
\end{document}
\end{myverbbox}

\begin{myverbbox}{\vbMax}
\documentclass{beamerikz}              % defines the document to be BeamerikZ

\begin{document}
\begin{bzFrame}                        % the first frame of that document
\bzOn{                                 % content to appear first
    \bzCenter{Welcome to \beamerikz!}  % a centred text
}
\bzOn{                                 % content to appear second
    \bzCenter{Welcome again, nice to see you!}
    \bzCenter{Hope that you'll stay longer :)}
}
\end{bzFrame}

\begin{bzFrame}                        % the second frame
\bzOn{                                 % content to appear first
    \bzCenter{Welcome again!}          % a centred text
}
\end{bzFrame}
\end{document}
\end{myverbbox}

\begin{myverbbox}{\vbBzOn}\bzOn{...}\end{myverbbox}
\begin{myverbbox}{\vbBzOnly}\bzOnly{...}\end{myverbbox}
\begin{myverbbox}{\vbBzOne}\bzOne{...}\end{myverbbox}
\begin{myverbbox}{\vbBzTwo}\bzTwo{...}\end{myverbbox}
\begin{myverbbox}{\vbBzIn}\bzIn{from}{to}{...}\end{myverbbox}

\begin{myverbbox}{\vbBzC}\bzCenter{...}\end{myverbbox}

\begin{bzFrame}%[show]
\bzIn{\bzS}{\bzS}{
	\iBzH
	\bzCode{\vbMin}
	\dBzH
}

\bzOn{
	\bzCenter[bzEB]{Minimal working example}
}

\bzIn{\bzS}{\bzS}{
	\path[fill=bzBlue!40, rounded corners=3pt] (\bzI+0.25, \bzH-3.5+0.25) rectangle ++(14.5, -2);
	\bzCode{\vbMid}
}

\bzOn{	
	\iBzH[10.0]	
	\bzLeft{To add more content, create a new \vbBzOn{} entry, as above.}
}

\bzOn{
	\bzLeft{Or add a new frame.}
	\dBzH[12]
	\path[fill=bzBlue!40, rounded corners=3pt] (\bzI+0.25, \bzH-6.5+0.25) rectangle ++(14.5, -2.5);
	\bzCode{\vbMax}
	\iBzH[12]
}

\bzOn{
	\bzLeft{\hint To avoid problems, put all your content (like \vbBzC)}
	\bzRight{inside \vbBzOn{} entries (or their variants)\ldots}
}
\end{bzFrame}

% =======================================================================================

\begin{myverbbox}{\vbBzM}bzM\end{myverbbox}
\begin{myverbbox}{\vbBzEvalInt}\bzEvalInt\end{myverbbox}
\begin{myverbbox}{\vbBzEvalFloat}\bzEval\end{myverbbox}

\begin{myverbbox}{\vbBzCenter}\bzCenter{...}\end{myverbbox}
\begin{myverbbox}{\vbBzLeft}\bzLeft{...}\end{myverbbox}
\begin{myverbbox}{\vbBzNext}\bzNext{...}\end{myverbbox}
\begin{myverbbox}{\vbBzPrev}\bzPrev{...}\end{myverbbox}
\begin{myverbbox}{\vbBzText}\bzText{...}\end{myverbbox}
\begin{myverbbox}{\vbBzRight}\bzRight{...}\end{myverbbox}
\begin{myverbbox}{\vbBzBox}\bzBox{...}\end{myverbbox}

\begin{bzFrame}%[show]
\bzOn{
	\bzCenter[bzEB]{Content}
}

\bzEval{\xs}{0.2}

\bzOn{
	\bzLeft{Use the following commands to draw your content into the frame:}
}

\iBzH[\xs]

\bzOn{
	\bzCenter{\vbBzCenter{} for centred text}
}

\iBzH[\xs]

\bzOn{
	\bzLeft{\vbBzLeft{} for text aligned next to the left edge}
}

\iBzH[\xs]

\bzOn{
	\bzText{\vbBzText{} for text written normally, from left, with an indent}
}

\iBzH[\xs]

\bzOn{
	\bzRight{\vbBzRight{} for text aligned to the right}
}

\iBzH[\xs]

\bzOn{
	\bzBox{\vbBzBox{} for content that exceeds the standard width of the slide and should wrap to another line (other commands may also allow that, see \vbBzM{} below\ldots)}
}
\end{bzFrame}

% =======================================================================================

\begin{myverbbox}{\vbBzItem}\bzItem{...}\end{myverbbox}
\begin{myverbbox}{\vbBzList}\bzList{...}\end{myverbbox}
\begin{myverbbox}{\vbBzL}\bzL\end{myverbbox}
\begin{myverbbox}{\vbzBzL}\zBzL\end{myverbbox}

\begin{myverbbox}{\vbBzItemDot}\bzItemDot\end{myverbbox}
\begin{myverbbox}{\vbzBzItemDot}\zBzItemDot\end{myverbbox}

\begin{myverbbox}{\vbRenewDot}\renewcommand{\bzItemDot}{$\circ$}\end{myverbbox}

\begin{bzFrame}%[show]
\bzOn{
	\bzCenter[bzEB]{Item and list}
}

\bzEval{\xs}{0.2}

\bzOn{
	\bzLeft{There are two ways to present lists of items:}
}

\bzOn{
	\bzItem{\vbBzItem{} for each of the consecutive items}
}

\bzOn{
	\bzItem{\vbBzItem{} \ldots}
}

\bzOn{
	\bzLeft{The big dot is drawn using \vbBzItemDot{} which you may:}
	\dBzH[0.2]
	\bzCode{\vbRenewDot}
}

\iBzH

\bzOn{
	\bzLeft{You can later reset it using:}
	\dBzH[0.2]
	\bzCode{\vbzBzItemDot}
}

\iBzH

\bzOn{
	\bzLeft{Or, if you need a (non-nested) list (the number of last item is stored in \vbBzL):}
}

\bzOn{
	\bzList{\vbBzList{} one entry}
}

\bzOn{
	\bzList{\vbBzList{} another}
}

\iBzH[\xs]

\bzOn{
	\bzLeft{To start your list again, use \vbzBzL{} (it zeroes \vbBzL), and then:}
}

\zBzL

\bzOn{
	\bzList{\vbBzList{} new first entry}
}

\bzOn{
	\bzList{\vbBzList{} second\ldots}
}

\bzOn{
	\bzLeft{\hint \vbBzEvalInt{} later in the manual will help you manage \vbBzL}
}
\end{bzFrame}

% =======================================================================================

\begin{myverbbox}{\vbBzTheorem}\bzTheorem{authors}{statement}\end{myverbbox}
\begin{myverbbox}{\vbBzTheoremS}\bzTheorem{...}{...}\end{myverbbox}
\begin{myverbbox}{\vbBzProof}\bzProof{...}\end{myverbbox}
\begin{myverbbox}{\vbBzQed}\bzQed\end{myverbbox}
\begin{myverbbox}{\vbBzLine}\bzLine\end{myverbbox}

\begin{bzFrame}%[show]
\bzOn{
	\bzCenter[bzEB]{More content\ldots}
}

\bzEval{\xs}{0.4}

\bzOn{
	\bzTheorem{Authors}{\vbBzTheorem{} to state a theorem (the statement can take multiple lines if needed)}
}

\iBzH

\bzOn{
	\bzLeft{\hint you can leave the authors empty if you don't want to specify them.}
}

\iBzH[0.5]

\bzOn{
	\bzLeft[bzM]{The same syntax works for: \hb{Lemma}, \hb{Fact}, \hb{Conjecture}, \hb{Definition}, \hb{Question}, \hb{Problem}, \hb{Proposition}, \hb{Corollary}, \hb{Exercise}, and \hb{Example}.}	
}

\iBzH[1.0]

\bzOn{
	\bzProof{\vbBzProof{} to provide a proof (again, it can be spread across multiple lines if needed)}
}

\iBzH

\bzOn{
	\bzLeft{Finally use \vbBzQed{} to conclude a proof:}
	\iBzH[-1]
	\bzQed
}

\iBzH[0.5]

\bzOn{
	\bzLeft{\vbBzLine{} draws a line:}
	\bzLine
}

\iBzH[0.5]

\end{bzFrame}

% =======================================================================================

\begin{myverbbox}{\vbBzBoxO}\bo{...}\end{myverbbox}
\begin{myverbbox}{\vbBzHL}\hl{...}\end{myverbbox}
\begin{myverbbox}{\vbBzHB}\hb{...}\end{myverbbox}
\begin{myverbbox}{\vbBzEL}\el{...}\end{myverbbox}
\begin{myverbbox}{\vbBzEB}\eb{...}\end{myverbbox}

\begin{myverbbox}{\vbBzTHL}\bzLeft[bzHL]{...}\end{myverbbox}
\begin{myverbbox}{\vbBzCHB}\bzCenter[bzHB]{...}\end{myverbbox}
\begin{myverbbox}{\vbBzRBO}\bzRight[bzBO]{...}\end{myverbbox}

\begin{bzFrame}%[show]
\bzOn{
	\bzCenter[bzEB]{Text style}
}

\bzEval{\xs}{0.3}

\bzOn{
	\bzLeft{Use the following commands to emphasise some text:}
}

\bzOn{
	\bzItem{\vbBzBoxO{} to only \bo{boldface} something (it works only with text, no math!)}
}

\bzOn{
	\bzItem{\vbBzHL{} to \hl{highlight} some text (including math: $a+\hl{b}+c$)}
}

\bzOn{
	\bzItem{\vbBzHB{} to \hb{highlight\&boldface} some text (again, no math!)}
}

\bzOn{
	\bzItem{\vbBzEL{} to \el{emphasize} some text (including math: $a+\el{b}+c$)}
}

\bzOn{
	\bzItem{\vbBzEB{} to \eb{emphasize\&boldface} some text (again, no math!)}
}

\iBzH[1-\xs]

\bzOn{
	\bzLeft[bzM]{The same suffixes with \Verb"bz" can be used for the standard content, e.g.:}
}

\bzOn{
	\bzLeft{\vbBzTHL{} }
	\bzNext[bzHL]{gives a highlighted left-aligned text}
}

\bzOn{
	\bzLeft{\vbBzCHB{}}
	\dBzH[0.25]
	\bzCenter[bzHB]{gives a highlighted\&boldfaced centred text}
}

\bzOn{
	\bzLeft{\vbBzRBO}
	\dBzH
	\bzRight[bzBO]{gives a boldfaced right-aligned text}
}

\iBzH[\xs]

\bzOn{
	\bzLeft[bzM]{\hint styles also apply to \vbBzBox, \vbBzTheoremS, \vbBzProof, etc}
}
\end{bzFrame}

% =======================================================================================


\begin{myverbbox}{\vbLeftPara}\bz____[...]{...}\end{myverbbox}

\begin{myverbbox}{\vbBzCenterSt}\bzCenter[scale=1.5, draw, inner sep=1mm, ellipse]{egg}\end{myverbbox}

\begin{bzFrame}%[show]
\bzOn{
	\bzCenter[bzEB]{\TikZ style}
}

\bzOn{
	\bzLeft{Each of the following commands (some are explained later):}
	\bzText[bzM]{\vbBzLeft, \vbBzText, \vbBzItem, \vbBzList, \vbBzRight, \vbBzCenter, \vbBzNext, \vbBzPrev, \vbBzBox, \vbBzTheoremS, \vbBzProof}
	\iBzH[3]
	\bzLeft{accepts an additional optional first parameter \vbLeftPara.}
}

\iBzH

\bzOn{
	\bzLeft{You can put any \TikZ style (applicable to a node) there.}
}

\bzOn{
	\bzLeft{The styles \Verb"bzBO", \Verb"bzHL", etc from the previous slide are examples of such styles.}
}

\iBzH[0.5]

\bzOn{
	\bzLeft{For instance}
	\bzText{\vbBzCenterSt}
}

\bzOn{
	\bzLeft{gives:}
	\bzCenter[scale=1.5, draw, inner sep=1mm, ellipse]{egg}
}
\end{bzFrame}

% =======================================================================================

\begin{myverbbox}{\vbWhole}
\documentclass{beamerikz}              % defines the document to be BeamerikZ

\sBzTitle{My title}                    % sets the title of the presentation
\sBzAuthor{Me Mysef}                   % sets the author (or authors)

\begin{document}
...\end{myverbbox}


\begin{myverbbox}{\vbTitle}\bzTitle\end{myverbbox}
\begin{myverbbox}{\vbAuthor}\bzAuthor\end{myverbbox}

\begin{myverbbox}{\vbBTitle}\sBzTitle{\bf The Title}\end{myverbbox}
\begin{myverbbox}{\vbUAuthors}\sBzAuthor{A. Guy, \underline{My Self}}\end{myverbbox}
\begin{myverbbox}{\vbNames}\bzNames{Reference, Relevance [2013]}\end{myverbbox}

\begin{bzFrame}%[show]
\bzOn{
	\bzCenter[bzEB]{Titles, authors, references}
}

\bzOn{
	\bzLeft{Control the footline of your frames using these macros (in the preamble):}
	\bzCode{\vbWhole}
}

\bzOn{
	\iBzH[4]
	\bzLeft{You can access these values using \vbTitle{} and \vbAuthor.}
}

\iBzH

\bzOn{
	\bzLeft{You may boldface your title: \vbBTitle}
}

\bzOn{
	\bzLeft{Or underline the speaker: \vbUAuthors}
}

\iBzH

\bzOn{
	\bzLeft{To refer to others' work, you may use:}
	\bzText{\vbNames}
	\bzCenter{\bzNames{Reference, Relevance [2013]}}
}
\end{bzFrame}

% =======================================================================================

\begin{myverbbox}{\vbCtext}\bzCtext\end{myverbbox}
\begin{myverbbox}{\vbCemph}\bzCemph\end{myverbbox}
\begin{myverbbox}{\vbChigh}\bzChigh\end{myverbbox}
\begin{myverbbox}{\vbCname}\bzCname\end{myverbbox}
\begin{myverbbox}{\vbCback}\bzCback\end{myverbbox}

\begin{myverbbox}{\vbRecolour}
\definecolor{mynewcolor}{RGB}{123,255,0}    % the RGB coordinates are between 0 and 255
\renewcommand{\bzCtext}{mynewcolor}
\end{myverbbox}

\begin{myverbbox}{\vbUseColour}\bzCenter{\textcolor{\bzCname}{this is also red}}\end{myverbbox}

\begin{bzFrame}%[show]
\bzOn{
	\bzCenter[bzEB]{Colours}
}

\newcommand{\showColor}[1]{
	\bzRight{(here it's \textcolor{#1}{{\bf #1}})}
}

\bzOn{
	\bzLeft{The layout is based on the following colours}
	\bzItem{\vbCtext{} used for text and drawings}
	\dBzH
	\showColor{\bzCtext}
}

\bzOn{
	\bzItem{\vbCemph{} used for titles}
	\dBzH
	\showColor{\bzCemph}
}

\bzOn{
	\bzItem{\vbChigh{} used for highlighting}
	\dBzH
	\showColor{\bzChigh}
}

\bzOn{
	\bzItem{\vbCname{} used for references}
	\dBzH
	\showColor{\bzCname}
}

\bzOn{
	\bzItem{\vbCback{} used for background}
	\dBzH
	\bzRight{(here it's \textcolor{bzWhite!60!bzBlack}{{\bf bzWhite}})}
}

\bzOn{
	\bzLeft{You can change each of these colours, by using (in the preamble):}
	\bzCode{\vbRecolour}
}

\iBzH[1.5]

\bzOn{
	\bzLeft{To make your presentation look uniform, use these colours consequently:}
	\bzCode{\vbUseColour}
	\iBzH
	\bzCenter{\textcolor{\bzCname}{this is also red}}
}

\bzOn{
	\bzLeft{\hint To choose your colours consistently, I recommend using:}
	\bzCenter{\href{https://coolors.co/}{https://coolors.co/}}
}
\end{bzFrame}

% =======================================================================================

\begin{myverbbox}{\vbPlain}
\begin{bzPlainFrame}
...
\end{bzPlainFrame}
\end{myverbbox}

\begin{myverbbox}{\vbTitleEx}
\begin{bzPlainFrame}
\bzOn{
    \bzCenter{\textcolor{\bzCtitle}{\bzTitle}}

    \bzCenter{\bzAuthor}
}

\bzOn{
    \bzCenter{\bf READY?}
}
\end{bzPlainFrame}
\end{myverbbox}


\begin{bzPlainFrame}%[show]
\bzOn{
	\bzCenter[bzEB]{Plain frames}
}

\bzOn{
	\bzLeft{Some frames (like the title one) need to be plain.}
}

\bzOn{
	\bzLeft{They go without the footline and are not counted in frame counter.}	
}

\bzOn{
	\bzRight{({\small like this one\ldots})}
}

\bzOn{
	\bzLeft{To create such a frame, use}	
	\bzCode{\vbPlain}
}

\bzOn{
	\iBzH[2]
	\bzLeft{The content of such a frame is generated in the same way as for other frames:}
	\bzCode{\vbTitleEx}
}
\end{bzPlainFrame}

% =======================================================================================

\tikzstyle{corarr}=[Circle-Latex]

\begin{myverbbox}{\vbBzCoords}\bzCoords\end{myverbbox}


\begin{bzFrame}%[show]
\bzCoords

\iBzH

\bzOn{
	\bzCenter[bzEB]{Coordinates}
}

\bzOn{
	\bzLeft{Each slide has the same coordinate system, with $(0, 0)$ here:\ \mbox{}}
	\draw (\bzLabel{\bzN}.north east) edge[corarr, bend right=30] (0,0);
}

\iBzH

\bzOn{
	\bzLeft{Horizontal axis ranges from $-10$ here:\ \mbox{ }}
	\draw (\bzLabel{\bzN}.north east) edge[corarr, bend right=10] (-10,\bzH+2);
}

\bzOn{
	\bzNext{ to $+10$ here:\ \mbox{}}
	\draw (\bzLabel{\bzN}.north east) edge[corarr, bend left=10] (+10,\bzH+2);
}

\iBzH

\bzOn{
	\bzLeft{Vertical axis decreases down to $-14.5$ here:\ \mbox{}}
	\draw (\bzLabel{\bzN}.north east) edge[corarr, bend left=10] (0,-14.5);
}

\iBzH

\bzOn{
	\bzLeft{That means that the point $(4, -6)$ is located here:\ \mbox{}}
	\draw (\bzLabel{\bzN}.north east) edge[corarr, bend right=10] (4,-6);
}

\iBzH

\bzOn{
	\bzLeft{The content is automatically clipped to the visible area, like those circles.}
	
	\path[fill=bzGray] (+9.5, -13.5) circle (1.5cm);
	\path[fill=bzGray] (-9.5, -13.5) circle (1.5cm);
	\path[fill=bzGray] (+9.5, 0) circle (1.5cm);
	\path[fill=bzGray] (-9.5, 0) circle (1.5cm);
}


\iBzH

\bzOn{
	\bzLeft{\hspace{3cm}\hint 1cm = 1unit of this coordinate system :)}
}

\bzOn{
	\bzLeft{\hspace{3cm}\hint \vbBzCoords{} shows the coordinate system}
}

\end{bzFrame}

% =======================================================================================

\begin{myverbbox}{\vbBezier}
\bzOn{
    \draw (-5, -3) .. controls ++(+3, +2) and ++(-3, -2) .. (+5, -3);
}\end{myverbbox}


\begin{myverbbox}{\vbNodes}
\bzOn{
	\node[bzL, draw] at (-8, -9) {aligns to left};
	\node[bzC, draw] at (+0, -9) {aligns to centre};
	\node[bzR, draw] at (+8, -9) {aligns to right};
}\end{myverbbox}


\begin{myverbbox}{\vbBaseline}
\bzOn{
    \node[bzL, draw] at (-8.0, -14) {car};
    \node[bzL, draw] at (-7.0, -14) {trunk};
    \node[bzL, draw] at (-5.0, -14) {yummy};
    \node[bzL, draw] at (-2.5, -14) {Tommy};
}
\end{myverbbox}

\begin{myverbbox}{\vbNodeTwoliner}
\node[bzC] at (0, -13) {spreads for\\ two lines};
\end{myverbbox}


\begin{bzFrame}%[show]
\bzCoords

\bzOn{
	\bzCenter[bzEB]{Nodes}
}

\bzOn{
	\bzLeft{You can use all the \TikZ machinery you like:}
	\bzCode{\vbBezier}
	\draw (-5, -4) .. controls ++(+3, +2) and ++(-3, -2) .. (+5, -4);
}

\iBzH[3.5]

\bzOn{
	\bzLeft{To write text, you may use styles \Verb"bzR", \Verb"bzC", and \Verb"bzL":}
	\bzCode{\vbNodes}
	\node[bzL, draw] at (-8, -9) {aligns to left};
	\node[bzC, draw] at (+0, -9) {lies centrally};
	\node[bzR, draw] at (+8, -9) {aligns to right};
	\node[circle, inner sep=1mm, fill=bzRed, draw=bzRed] at (-8,-9) {};
	\node[circle, inner sep=1mm, fill=bzRed, draw=bzRed] at (+0,-9) {};
	\node[circle, inner sep=1mm, fill=bzRed, draw=bzRed] at (+8,-9) {};
}

\iBzH[3.5]

\bzOnly{
	\bzLeft{\hint in fact \vbBzLeft, \vbBzCenter, \ldots are based on these styles!}
}

\dBzH

\bzOn{
	\bzLeft{What is important, is that the baseline of text is kept:}
	\bzCode{\vbBaseline}
	\node[bzL, draw] at (-8.0,  -14) {car};
	\node[bzL, draw] at (-7.0,  -14) {trunk};
	\node[bzL, draw] at (-5,  -14) {yummy};
	\node[bzL, draw] at (-2.5,  -14) {Tommy};
}
\end{bzFrame}

% =======================================================================================

\begin{myverbbox}{\vbUnder}\bzUnder{...}\end{myverbbox}

\begin{myverbbox}{\vbUnderEx}
\bzOn{
    \path[fill=bzRed] (5, -10) rectangle (9, -12);
}

\bzUnder{
    \bzOn{
        \path[fill=bzBlue] (6, -9) rectangle (8, -13);
    }
}
\end{myverbbox}

\begin{bzFrame}%[show]
\bzOn{
	\bzCenter[bzEB]{Underlayer}
}

\bzOn{
	\bzLeft{You can place some content under the existing content.}
}

\bzOn{
	\bzLeft{Use the following commands to draw on the under layer:}
}

\bzOn{
	\bzCode{\vbUnderEx}
	\path[fill=bzRed] (5, -10) rectangle (9, -12);
}

\bzUnder{
	\bzOn{
		\path[fill=bzBlue] (6, -9) rectangle (8, -13);
	}
}

\iBzH[5]

\bzOn{
	\bzLeft{\hint \vbUnder{} must contain \vbBzOn{}, not the other way round!}
}
\end{bzFrame}
% =======================================================================================
	
\begin{myverbbox}{\vbBzHStep}
\bzHStep
\end{myverbbox}

\begin{myverbbox}{\vbsBzHStep}
\sBzHStep{2.0}
\end{myverbbox}

\begin{myverbbox}{\vbsBzHStepOne}
\sBzHStep{1.0}
\end{myverbbox}

\begin{myverbbox}{\vbzBzHStep}
\zBzHStep
\end{myverbbox}


\bzEval{\bzHStp}{1.0}

\begin{bzFrame}%[show]
\bzCoords

\bzOn{
	\bzCenter[bzEB]{Vertical spacing}
}

\bzOn{
	\bzLeft{Each consecutive command like \vbBzLeft{} spreads text vertically\ldots}
	\bzLeft{\ldots like these two.}
}

\bzOn{
	\bzLeft{The vertical step is stored in the counter \vbBzHStep.}
	\bzLeft{By default, its value is 1.0.}
}

\bzOn{
	\bzLeft{To control the vertical spacing between the lines of your text, use:}
	\bzCode{\vbsBzHStep}
	\iBzH
	\sBzHStep{2.0}
}

\bzOn{
	\bzLeft{Which changes the spacing}
	\bzLeft{as seen}
	\bzLeft{here!}
}

\bzOn{
	\bzLeft{\vbzBzHStep{} is equivalent to \vbsBzHStepOne, i.e. it restores the value to 1.0.}
}
\end{bzFrame}

% =======================================================================================

\begin{myverbbox}{\vbBzH}\bzH\end{myverbbox}
\begin{myverbbox}{\vbsBzH}\sBzH{4.0}\end{myverbbox}
\begin{myverbbox}{\vbiBzH}\iBzH\end{myverbbox}
\begin{myverbbox}{\vbiBzHp}\iBzH[0.8]\end{myverbbox}
\begin{myverbbox}{\vbdBzH}\dBzH\end{myverbbox}
\begin{myverbbox}{\vbdBzHp}\dBzH[1.3]\end{myverbbox}
\begin{myverbbox}{\vbzBzH}\zBzH\end{myverbbox}

\begin{myverbbox}{\vbConRel}
    \draw (-4, \bzH) rectangle (+4, \bzH-1);
\end{myverbbox}

\begin{bzFrame}%[show]
\bzCoords

\bzOn{
	\bzCenter[bzEB]{Vertical alignment}
}

\bzOn{
	\bzLeft{There is a counter called \vbBzH, controlling the \emph{height} of the content.}
}

\bzOn{
	\bzText{Its value is now \bzH}
}

\bzOn{
	\bzText{and now \bzH}
}

\bzOn{
	\bzLeft{Each new line decreases it by \vbBzHStep{} (usually equal 1.0) so that now it's \bzH.}
}

\bzOn{
	\bzLeft{You should place {\bf all} your content relatively to \vbBzH(!):}
	\bzCode{\vbConRel}
	\iBzH
	\bzRight{$\vbBzH=\bzH$}
	\dBzH
    \draw (-4, \bzH) rectangle (+4, \bzH-1);
}

\bzOn{
	\iBzH[2]
	\bzRight{$\vbBzH=\bzH$}
	\dBzH
	\bzLeft{To get some vertical space, use:}
	\bzRight{$\vbBzH=\bzH$}
	\dBzH
	\bzText{\vbiBzH}
}

\bzOn{
	\bzRight{$\vbBzH=\bzH$}
	\dBzH
	\bzLeft{That \emph{increases} (in fact decreases\ldots) \vbBzH{} by \vbBzHStep}
}

\bzOn{
	\bzLeft{You can control the amount of space by an optional parameter:}
	\dBzH
	\bzRight{$\vbBzH=\bzH$}
	\bzText{\vbiBzHp}
	\dBzH
	\bzRight{$\vbBzH=\bzH$}
}
	
\bzOn{
	\dBzH
	\iBzH[0.8]
	\bzRight{$\vbBzH=\bzH$}
	\dBzH
	\bzLeft{Which \emph{increases} (in fact decreases\ldots) \vbBzH{} by $0.8$}
}
\end{bzFrame}

% =======================================================================================


\begin{myverbbox}{\vbArith}
\newcommand{\x}{1.6}
\iBzH[(\x+1)*0.5+2]
\end{myverbbox}
	
\begin{bzFrame}%[show]
\bzOn{
	\bzCenter[bzEB]{Vertical alignment --- summary}
}

\bzOn{
	\bzLeft{Available commands:}
}

\bzOn{
	\bzItem{\vbBzH{} --- the counter}
}

\bzOn{
	\bzItem{\vbsBzH{} --- set the value of the counter}
}

\bzOn{
	\bzItem{\vbiBzH{} --- go down one line (i.e. decrease \vbBzH{} by \vbBzHStep, 1.0 by default)}
}

\bzOn{
	\bzItem{\vbiBzHp{} --- go down by $0.8$}
}

\bzOn{
	\bzItem{\vbdBzH{} --- go up one line (i.e.~\emph{increase} \vbBzH{} by \vbBzHStep)}
}

\bzOn{
	\bzItem{\vbdBzHp{} --- go up $1.3$ of a line}
}

\bzOn{
	\bzItem{\vbzBzH{} --- reset \vbBzH{} to $0$ (i.e.~\emph{zero} \vbBzH)}
}

\iBzH[0.5]

\bzOn{
	\bzLeft{All these commands accept arithmetic, like:}
	\bzCode{\vbArith}
}

\iBzH[2.0]

\bzOn{
	\bzLeft{\hint you may put such commands both {\bf inside} and {\bf outside} of \vbBzOn}
	\dBzH[0.2]
	\bzCenter{({\small all the commands above the given one do affect it})}
}
\end{bzFrame}

% =======================================================================================

\begin{myverbbox}{\vbBzI}\bzI\end{myverbbox}
\begin{myverbbox}{\vbsBzI}\sBzI{5.0}\end{myverbbox}
\begin{myverbbox}{\vbiBzI}\iBzI\end{myverbbox}
\begin{myverbbox}{\vbiBzIp}\iBzI[2.8]\end{myverbbox}
\begin{myverbbox}{\vbdBzI}\dBzI\end{myverbbox}
\begin{myverbbox}{\vbdBzIp}\dBzI[1.3]\end{myverbbox}
\begin{myverbbox}{\vbzBzI}\zBzI\end{myverbbox}

\begin{myverbbox}{\vbConRelT}
    \draw (-4, \bzH) rectangle (+4, \bzH-1);
\end{myverbbox}

\begin{bzFrame}%[show]
\bzCoords

\bzOn{
	\bzCenter[bzEB]{Horizontal alignment}
}

\bzOn{
	\bzLeft{There is a counter called \vbBzI, controlling the \emph{indentation} of the content.}
}

\bzOn{
	\bzLeft{Using \vbBzLeft{} when $\vbBzI=\bzI$}
}

\iBzI[5]

\bzOn{
	\bzLeft{now \vbBzLeft{} again but $\vbBzI=\bzI$}
}

\iBzI[8]

\bzOn{
	\bzLeft{and now $\vbBzI=\bzI$}
}

\dBzI[13]

\bzOn{
	\bzText{\vbBzText{} is always $1$ to the right (now $\vbBzI=-10$ again)}
}

\iBzH[0.3]

\bzOn{
	\bzCenter{\vbBzCenter{} is not affected}
}

\iBzH[0.3]

\zBzI
\iBzI[5]

\bzOn{
	\bzRight{\vbBzRight{} is affected by $-\vbBzI$ (now $\vbBzI=\bzI$)}
}

\zBzI

\iBzH[0.3]

\bzOn{
	\bzLeft{Available commands:}
}

\bzOn{
	\bzItem{\vbsBzI{} --- set the value of the counter}
}

\bzOn{
	\bzItem{\vbiBzI{} --- increase indent by $1$ (+parametrised \vbiBzIp)}
}

\bzOn{
	\bzItem{\vbdBzI{} --- decrease indent by $1$ (+parametrised \vbdBzIp)}
}

\bzOn{
	\bzItem{\vbzBzI{} --- reset \vbBzI{} to $0$}
}

\bzOn{
	\bzLeft{\hint you may modify \vbBzI{} {\bf outside} \vbBzOn}
}
\end{bzFrame}


% =======================================================================================
	
\begin{myverbbox}{\vbNL}\\\end{myverbbox}

\begin{myverbbox}{\vbBzMEx}
\iBzI[3.5]
\bzLeft[bzM]{Text that is longer than a line and the style bzM brakes it automatically}
\end{myverbbox}

\begin{myverbbox}{\vbBzMSet}
\bzLeft[bzM=5.5, draw]{Text of width 5.5 with automatic line breaks}
\end{myverbbox}

\begin{myverbbox}{\vbBzHStep}
\bzHStep
\end{myverbbox}

\begin{myverbbox}{\vbsBzHStep}
\sBzHStep{1.25}
\end{myverbbox}

\begin{bzFrame}%[show]
\bzCoords

\bzOn{
	\bzCenter[bzEB]{Multiline nodes}
}

\bzOn{
	\bzLeft{Standard \vbBzLeft, etc commands do not break lines automatically, as here where\ldots}
}

\bzOn{
	\iBzH[1]
	\node[circle, inner sep=1mm, fill=bzRed, draw=bzRed] at (\bzI+1,\bzH) {};
	\bzText[draw]{You can manually break lines using the newline syntax \vbNL\\ however, the node is then aligned vertically to the \hb{last} line of text}
}

\bzOn{
	\bzLeft{To allow automatic line breaks, use the \hl{multiline} style \Verb"bzM":}
	\bzCode{\vbBzMEx}
	\iBzH[1.5]
	\iBzI[3.5]
	\node[circle, inner sep=1mm, fill=bzRed, draw=bzRed] at (\bzI,\bzH) {};
	\bzLeft[bzM]{Text which is longer than a line and the style \Verb"bzM" brakes it automatically}
	\zBzI
	\iBzH
}

\bzOn{
	\bzLeft{The same works for other nodes, like \vbBzRight, \vbBzCenter, etc}
}

\iBzH[0.5]

\bzOn{
	\bzLeft[bzM]{\hint The vertical spacing between the lines is controlled by \vbBzHStep!}
}

\iBzH[0.5]

\bzOn{
	\bzLeft[bzM]{\hint \beamerikz is not aware of the actual height of the node, so increase \vbBzH{} manually!}
}

\iBzH[0.5]

\bzOn{
	\iBzH[1]
	\bzLeft{\hint \Verb"bzM" aligns vertically to the \hl{first} line of text and allows explicit newlines \vbNL}
}
\end{bzFrame}

% =======================================================================================

\begin{bzFrame}%[show]
\bzCoords

\bzOn{
	\bzCenter[bzEB]{Multiline nodes --- width}
}

\bzOn{
	\bzLeft[bzM]{By default, the width of the text is adjusted so that it spreads until the end of the working area $(-10,\ldots,+10)$ (or $(-13.0,\ldots,+13.0)$ in \texttt{wide} mode).}
	\iBzH
	\bzLeft{In \vbBzCenter, margins of width given by \vbBzI{} are left on both sides.}
}

\bzOn{
	\sBzI{-3}
	\bzLeft[bzM, draw]{Text aligned to left by \vbBzLeft{} and being broken to a new line.}
	\iBzH
	\bzRight[bzM, draw]{Text aligned to right by \vbBzRight{} and being broken to a new line.}
	\iBzH
	\bzCenter[bzM, draw]{Text centered via \vbBzCenter{} and broken to new lines.}
	\iBzH
}

\iBzH[1.25]
\zBzI

\bzOn{
	\bzLeft{You can set your own width of text, by using \Verb"bzM=<num>" like here}
	\bzCode{\vbBzMSet}
	\iBzH
	\node[circle, inner sep=1mm, fill=bzRed, draw=bzRed] at (\bzI,\bzH) {};
	\bzLeft[bzM=5.5, draw]{Text of width 5.5 with automatic line breaks}
}
\end{bzFrame}

% =======================================================================================


\begin{myverbbox}{\vbStdMath}$...$\end{myverbbox}
\begin{myverbbox}{\vbDisMath}\[...\]\end{myverbbox}
\begin{myverbbox}{\vbBzEq}\bzEq{...}\end{myverbbox}
\begin{myverbbox}{\vbDisMathEx}\bzCenter[bzM]{\[\sum_{i=1}^n i=\frac{n\cdot(n+1)}{2}\]}\end{myverbbox}
\begin{myverbbox}{\vbBzEqEx}\bzRight{$\bzEq{\sum_{i=1}^n i=\frac{n\cdot(n+1)}{2}}$}\end{myverbbox}
\begin{myverbbox}{\vbBzEqMlEx}
\bzCenter{$\bzEq{
  x_0 &= x_1 = 0 &\quad\text{(1)}\\
  x_{n+2} &= x_{n+1}+x_n &\quad\text{(2)}
}$}
\end{myverbbox}
\begin{myverbbox}{\vbAlign}align*\end{myverbbox}

\begin{bzFrame}%[show]
\bzOn{
	\bzCenter[bzEB]{Mathematics}
}
	
\bzOn{
	\bzItem{Standard math mode \vbStdMath{} is allowed everywhere, like here $x=1$.}
}
	
\bzOn{
	\bzItem{To use displayed mathematics \vbDisMath{}, the option \vbBzM{} is required:}
	\bzCode{\vbDisMathEx}
	\bzCenter[bzM]{\[\sum_{i=1}^n i=\frac{n\cdot(n+1)}{2}\]}
	\iBzH[2]
	\bzText[bzM]{\ldots but \vbDisMath{} introduces additional vertical space. \hb{Better avoid it.}}
	\bzText{Similarly \vbAlign{} is also not suggested\ldots}
}
		
\bzOn{
	\bzItem{Instead, use \vbBzEq{} as below:}
	\bzCode{\vbBzEqEx}
	\bzRight{$\bzEq{\sum_{i=1}^n i=\frac{n\cdot(n+1)}{2}}$}
}

\bzOn{
	\bzItem{It allows formatting as in \vbAlign:}
	\bzCode{\vbBzEqMlEx}
	\iBzH[2.0]
	\bzCenter{$\bzEq{
		x_0&=x_1=0&\quad\text{(1)}\\
		x_{n+2}&=x_{n+1}+x_n&\quad\text{(2)}
	}$}
}
\end{bzFrame}

% =======================================================================================

\begin{myverbbox}{\vbBzGraphics}\node[bzG] at (5,-7) {\includegraphics[width=6cm]{logo.png}};\end{myverbbox}

\begin{myverbbox}{\vbBzCen}(5,-7)\end{myverbbox}
\begin{myverbbox}{\vbBzWid}6cm\end{myverbbox}

\begin{bzFrame}%[show]
\bzCoords

\bzOn{
	\bzCenter[bzEB]{Graphics}
}

\bzOn{
	\path[fill=\bzCtext, opacity=0.3] (1, -3) rectangle ++(8, -8);
	\bzLeft{To include graphics, you may use:}
}

\bzOn{
	\bzCenter{\vbBzGraphics}
	\node[bzG] at (5, -7) {\includegraphics[width=6cm]{logo.png}};
}

\iBzH

\bzOn{
	\bzItem{The centre is at \vbBzCen}
	\path ($(\bzLabel{\bzN}.east)+(0.25,0)$) edge[draw, Circle-Latex, bend left] (5,-7);
}

\bzOn{
	\bzItem{Width is \vbBzWid{} = 6 units}
	\draw (2,-10) edge[bzBraceU] node{6 units} ++(6,0);
}

\bzOn{
	\bzItem{Height is proportional to the ratio}
	\draw (8,-10) edge[bzBraceR] node{6 units} ++(0,6);
}

\bzOn{
	\bzItem{There are no margins or spacing}
}

\end{bzFrame}

% =======================================================================================

\begin{myverbbox}{\vbEval}
\bzEvalInt{\x}{7*(4+\bzH)+0.25}
\bzEval{\y}{6/(2-\bzH)}
\end{myverbbox}

\begin{myverbbox}{\vbX}\x\end{myverbbox}
\begin{myverbbox}{\vbY}\y\end{myverbbox}
\begin{myverbbox}{\vbBzEvalFloatEx}\bzEval{\curH}{\bzH}\end{myverbbox}
\begin{myverbbox}{\vbCurH}\curH\end{myverbbox}
\begin{myverbbox}{\vbDrawArRef}\draw (0, \bzH-1) edge[Circle-Latex, bend right=40] (9, \curH);\end{myverbbox}

\begin{bzFrame}%[show]
\bzOn{
	\bzCenter[bzEB]{Arithmetic}
}

\bzOn{
	\bzLeft{To perform complex computations, you may use the following two functions:}
	
	\bzEvalInt{\x}{7*(4+\bzH)+0.25}
	\bzEval{\y}{6/(2-\bzH)}
	
	\bzCode{\vbEval}
}

\bzOn{
	\bzRight{$\vbBzH=\bzH$}
	\dBzH
}

\iBzH[1.5]

\bzOn{
	\bzLeft{They compute {\bf immediately}, using {\bf current} values of the variables.}
}

\bzOn{
	\bzLeft{The first stores the value as a rounded integer, the second is floating point.}
}

\bzOn{
	\bzLeft{Thus, we get $\vbX=\x$ and $\vbY=\y$, even though now $\vbBzH=\bzH$.}
}

\iBzH[0.5]

\bzOn{
	\bzLeft{\hint do not mix \vbBzEvalInt{} with \vbBzEvalFloat!}
}

\iBzH[0.5]

\bzOn{
	\bzEval{\curH}{\bzH}
	\bzLeft{You may use \vbBzEvalFloatEx{} to store in the variable \vbCurH}
	\bzRight{the current value of \vbBzH{} for later use!}
}

\bzOn{
	\draw (0, \bzH-2) edge[Circle-Latex, bend right=40] (9, \curH);
	\bzLeft{And then use:}
	\bzLeft{\vbDrawArRef}
	\bzRight{to use that value!}
}
\end{bzFrame}

% =======================================================================================

\begin{myverbbox}{\vbScript}
\newcommand{\putDot}[2]{
  \bzEval{\x}{#1}
  \bzEval{\r}{#2}
  \node[draw, circle, rotate=\r] at (\x, \bzH) {Hi!};
}

\bzOn{
  \foreach \i in {0,...,9} {
    \putDot{\bzI+2*\i+1}{\i*20}
  }
}
\end{myverbbox}

\begin{myverbbox}{\vbBzFrame}\begin{bzFrame}...\end{bzFrame}\end{myverbbox}


\newcommand{\putDot}[2]{
	\bzEval{\x}{#1}
	\bzEval{\r}{#2}
	\node[draw, circle, rotate=\r] at (\x, \bzH) {Hi!};
}

\begin{bzFrame}%[show]
\bzOn{
	\bzCenter[bzEB]{Arithmetic and scripting}
	\iBzH[0.5]
}

\bzOn{
	\bzLeft[bzM]{To avoid syntactic problems, use \vbBzEvalFloat{} to give names to parameters of your macros.}
	\iBzH
}

\bzOn{
	\bzCode{\vbScript}
	\iBzH[6.0]
}

\bzOn{
	\bzLeft{Gives:}
	\foreach \i in {0,...,9} {
		\putDot{\bzI+2*\i+1}{\i*20}
	}
}

\bzOn{
	\iBzH[2]
	\bzLeft[bzM]{\hint For readability, you can put your macros outside \Verb"bzFrame" environment}
}
\end{bzFrame}

% =======================================================================================

\begin{myverbbox}{\vbBzS}\bzS\end{myverbbox}
\begin{myverbbox}{\vbsBzS}\sBzS{4}\end{myverbbox}
\begin{myverbbox}{\vbiBzS}\iBzS\end{myverbbox}
\begin{myverbbox}{\vbiBzSp}\iBzS[2]\end{myverbbox}
\begin{myverbbox}{\vbdBzS}\dBzS\end{myverbbox}
\begin{myverbbox}{\vbdBzSp}\dBzS[1]\end{myverbbox}
\begin{myverbbox}{\vbzBzS}\zBzS\end{myverbbox}

\newcommand{\doS}[1][0]{
\bzEvalInt{\oneMore}{\bzS+#1}
\bzIn{\oneMore}{\oneMore}{
	\node[bzL] at (7, 0) {$\vbBzS=\oneMore$};
}
}

\begin{bzFrame}%[show]
\bzOn{
	\bzCenter[bzEB]{Slide counter --- basics}
}

\doS

\bzOn{
	\bzLeft{To control the flow of time within a frame, a counter \vbBzS{} is used.}
}

\doS

\bzOn{
	\bzLeft{Each \vbBzOn{} shows a new content, increasing \vbBzS{}.}
}

\doS

\bzOne{
	\bzLeft{\vbBzOne{} shows its content for one slide and disappears.}
}

\doS
		
\bzOn{
	\bzLeft{The next \vbBzOn{} appears at the moment when \vbBzOne{} disappears.}
}

\doS

\bzTwo{
	\bzLeft{\vbBzTwo{} shows its content for two slides and disappears.}
}

\doS
		
\bzOn{
	\bzLeft{The next \vbBzOn{} appears afterwards.}
}

\doS
		
\bzOn{
	\bzLeft{And only the second \vbBzOn{} appears when \vbBzTwo{} disappears.}
}

\doS

\doS[1]

\bzOnly{
	\bzLeft{\vbBzOnly{} appears for one slide and takes one more to disappear.}
}

\doS
		
\bzOn{
	\bzLeft{The next \vbBzOn{} appears after one more click.}
}

\doS

\bzOn{
	\iBzH[0.5]
	\bzLeft{\hint Some package options display the current value of \vbBzS{}, see }
	\bzRight{\Verb"draft" and \Verb"brief" later on}
}
\end{bzFrame}

% =======================================================================================

\begin{myverbbox}{\vbBzEvalIntS}\bzEvalInt{\curS}{\bzS}\end{myverbbox}
\begin{myverbbox}{\vbCurS}\curS\end{myverbbox}
\begin{myverbbox}{\vbBzInEx}\bzIn{\curS+2}{\curS+7}{...}\end{myverbbox}

\begin{bzFrame}%[show]
\doS

\bzOn{
	\bzCenter[bzEB]{Slide counter --- \vbBzIn}
}

\doS

\bzOn{
	\bzLeft{Because of the order of drawing (or other reasons), you may need}
	\bzRight{to manually handle when content (dis)appears.}
}

\doS

\bzOn{
	\bzEvalInt{\curS}{\bzS}
	\bzLeft{First, use \vbBzEvalIntS{} to store the current value of \vbBzS.}
}

\doS

\bzOn{
	\bzLeft{Then, use \vbBzIn{} to control when some content is visible.}
}

\doS

\bzIn{\curS+2}{\curS+7}{
	\bzCode{\vbBzInEx}
	\bzRight{$\vbCurS=\curS$}
}

\bzOn{
	\bzLeft{Rules:}
}

\doS

\bzOn{
	\bzList{If both arguments are equal, then the content lasts one slide.}
}

\doS

\bzOn{
	\bzList{If the first argument is empty, it's visible from the beginning.}
}

\doS

\bzOn{
	\bzList{If the second argument is empty, it's visible until the end of the frame.}
}

\doS

\bzOn{
	\bzList{You can do any (integer) arithmetic within the arguments.}
}

\doS

\bzOn{
	\bzList{You can use \vbBzS{} within the arguments --- it takes its current value.}
}

\doS

\bzOn{
	\bzList{\vbBzIn{} does not modify \vbBzS{}.}
}

\doS

\bzOn{
	\bzList{You should ensure that \Verb"from" $\leq$ \Verb"to".}
}

\doS

\bzOn{
	\bzList{\vbBzS{} starts with $1$.}
}
\end{bzFrame}

% =======================================================================================

\begin{bzFrame}%[show]
\bzOn{
	\bzCenter[bzEB]{Slide counter --- summary}
}

\bzOn{
	\bzLeft{Available commands:}
}

\bzOn{
	\bzItem{\vbsBzS{} --- set slide counter value}
}

\bzOn{
	\bzItem{\vbiBzS{} --- increase slide counter by $1$ (+parametrised \vbiBzSp)}
}

\bzOn{
	\bzItem{\vbdBzS{} --- decrease slide counter by $1$ (+parametrised \vbdBzSp)}
}

\bzOn{
	\bzItem{\vbzBzS{} --- reset \vbBzS{} to $1$}
}

\iBzH[0.5]

\bzOn{
	\bzLeft{\hint if you increase \vbBzS{} using \vbiBzS{} without showing any more content,}
	\bzRight{it will not make your slide last longer!}
}

\iBzH[0.5]

\bzOn{
	\bzLeft{\hint if possible, avoid guessing correct numbers for \vbBzIn,}
	\bzRight{use a variant of \vbBzEvalIntS{} instead!}
}

\iBzH[0.5]

\bzOn{
	\bzLeft{\hint remember that the content overlays each other.}
	\bzText{The order comes from the order of entries in the source file,}
	\bzRight{not from the values of \vbBzS.}
}
\end{bzFrame}

% =======================================================================================

\begin{myverbbox}{\vbBzN}\bzN\end{myverbbox}

\begin{myverbbox}{\vbBzEvalIntN}\bzEvalInt{\curN}{\bzN}\end{myverbbox}
\begin{myverbbox}{\vbCurN}\curN\end{myverbbox}

\begin{myverbbox}{\vbNeEx}\bzNext{\bf I'm next}[\curN]\end{myverbbox}
\begin{myverbbox}{\vbPrEx}\bzPrev{\bf I'm prev}[\curN]\end{myverbbox}

\begin{myverbbox}{\vbBzPrevSt}\bzPrev[draw]{...}\end{myverbbox}
	
\begin{bzFrame}%[show]
\bzOn{
	\bzCenter[bzEB]{Named nodes}
}

\bzOnly{
	\bzLeft{These commands create \TikZ nodes:}
	\bzText{\vbBzLeft, \vbBzText, \vbBzItem, \vbBzList,}
	\bzText{\vbBzCenter, \vbBzRight, \vbBzNext, \vbBzBox.}
}

\dBzH[3]

\bzOn{
	\bzLeft{You can append new content to the last node using \vbBzNext:}
}

\bzOn{
	\bzLeft{\vbBzLeft}
}

\bzOn{
	\bzNext{\vbBzNext}
}

\bzOn{
	\bzNext{\vbBzNext}
}

\bzOn{
	\bzLeft{You can also attach content to the left (once):}
}

\bzOn{
	\bzCenter{\vbBzCenter}
}

\bzOn{
	\bzPrev{\vbBzPrev}
}

\bzOn{
	\bzNext{\vbBzNext}
}

\bzOn{
	\bzNext{\vbBzNext}
}

\bzOn{
	\bzLeft{\hint to surround a node:}
	\bzCenter{first use \vbBzPrev{} and then \vbBzNext!}
}

\bzOn{
	\bzLeft{This is controlled by a counter named \vbBzN.}
}

\bzOn{
	\bzLeft{\hint you should never modify the value of that counter yourself!}
}

\bzOn{
	\bzLeft{You can store the number of the last node using:}
	\bzCenter{\vbBzEvalIntN}
	\bzEvalInt{\curN}{\bzN}
}

\bzOn{
	\bzLeft{You can later append content to that node using parametrised versions:}
}

\bzOn{
	\bzNext{\bf I'm next}[\curN]
	\bzText{\vbNeEx}
}

\bzOn{
	\bzPrev{\bf I'm prev}[\curN]
	\bzText{\vbPrEx}	
}

\bzOn{
	\bzLeft{\hint nodes created by \vbBzPrev{} do not have labels!}
}
\end{bzFrame}

% =======================================================================================

\begin{myverbbox}{\vbLargEqA}
\bzCenter[scale=1.3]{$=$}
\bzPrev[scale=1.3]{$\int_a^b f'(x)\ $}
\bzNext[scale=1.3]{$\ f(x)\Big|_a^b$}
\end{myverbbox}


\begin{myverbbox}{\vbLargEqB}
\bzCenter[scale=1.3]{$\leq$}
\bzPrev[scale=1.3]{$\Big|\int_a^b f(x)\Big|\ $}
\bzNext[scale=1.3]{$\ \int_a^b \big|f(x)\big|$}
\end{myverbbox}

\begin{myverbbox}{\vbToL}[bzR]\end{myverbbox}
\begin{myverbbox}{\vbToR}[bzL]\end{myverbbox}
\begin{myverbbox}{\vbToC}[bzC]\end{myverbbox}

\begin{myverbbox}{\vbToLCR}
\node[bzR] at (5-1, \bzH) {$(f(g(x))'$};
\node[bzC] at (5+0, \bzH) {$=$};
\node[bzL] at (5+1, \bzH) {$(f'(g(x))\cdot g'(x)$};
\end{myverbbox}


\begin{myverbbox}{\vbToLRMatch}
\node[bzR] at (0, \bzH) {deoxyribo}; \node[bzL] at (0, \bzH) {nucleic acid};
\end{myverbbox}

\begin{bzFrame}%[show]
\bzCoords

\bzOn{
	\bzCenter[bzEB]{Named nodes --- hints and tricks}
}

\bzOn{
	\bzLeft{\hint you can style these nodes: \vbBzPrevSt}
}

\bzOn{
	\bzLeft{To create an aligned large mathematical equation, you can use:}
}

\bzOn{
	\dBzI[0.5]
	\bzCode{\vbLargEqA}
	\iBzI[9.25]
	\bzCode{\vbLargEqB}
	\zBzI
}

\iBzH[2.5]

\bzOn{
	\bzCenter[scale=1.3]{$=$}
	\bzPrev[scale=1.3]{$\int_a^b f'(x)\ $}
	\bzNext[scale=1.3]{$\ f(x)\Big|_a^b$}
}

\iBzH[0.75]

\bzOn{
	\bzCenter[scale=1.3]{$\leq$}
	\bzPrev[scale=1.3]{$\Big|\int_a^b f(x)\Big|\ $}
	\bzNext[scale=1.3]{$\ \int_a^b \big|f(x)\big|$}
}

\iBzH[0.5]

\bzOn{
	\bzLeft{However, a similar effect can be also achieved via \vbToL, \vbToR, and \vbToC:}
}

\bzOn{
	\dBzI[0.5]
	\bzCode{\vbToLCR}
	\zBzI
}

\iBzH[0.5]

\bzOn{
	\node[bzR] at (5-1, \bzH) {$(f(g(x))'$};
	\node[bzC] at (5+0, \bzH) {$=$};
	\node[bzL] at (5+1, \bzH) {$(f'(g(x))\cdot g'(x)$};
}

\iBzH[1.75]

\bzOn{
	\bzLeft{\hint you can break a word using \vbToL{} and \vbToR{} (and the nodes match):}
}

\bzOn{
	\dBzI[0.5]
	\bzCode{\vbToLRMatch}
	\zBzI
}

\iBzH

\bzOn{
	\node[bzR] at (0, \bzH) {deoxyribo};
}

\bzOn{
	\node[bzL] at (0, \bzH) {nucleic acid};
}
\end{bzFrame}

% =======================================================================================

\begin{myverbbox}{\vbRefNode}
\bzCenter{Referable node!}
\bzEvalInt{\nodeN}{\bzN}
\end{myverbbox}

\begin{myverbbox}{\vbBzLabelN}\bzLabel{\nodeN}\end{myverbbox}

\begin{myverbbox}{\vbBzLabelDraw}\draw (7, \bzH) edge[Circle-Latex, bend right=50] (\bzLabel{\nodeN}.mid east);\end{myverbbox}


\begin{myverbbox}{\vbBzLabelDrawT}\draw (8, \bzH) edge[Circle-Latex, bend right=50] ($(\bzLabel{\nodeN}.east)+(4,0)$);\end{myverbbox}
\begin{myverbbox}{\vbBzLabelDrawU}\coordinate (base) at ($(\bzLabel{\nodeN}.east)+(0,-0.5)$);
\draw
    (\bzLabel{\nodeN}.west |- base)
    edge[bzBraceU] node{the one}
    (base -| \bzLabel{\nodeN}.east);\end{myverbbox}

\begin{bzFrame}%[show]
\bzOn{
	\bzCenter[bzEB]{Named nodes --- explicit references}
}

\bzOn{
	\bzLeft{Let's create a referable node:}
}

\bzOn{
	\bzCode{\vbRefNode}
}

\iBzH[0.5]

\bzOn{
	\bzCenter{Referable node!}
	\bzEvalInt{\nodeN}{\bzN}
}

\iBzH[1.0]

\bzOn{
	\bzLeft{It's label can be obtained by \vbBzLabelN:}
}

\bzOn{
	\bzCode{\vbBzLabelDraw}
	\draw (7, \bzH) edge[corarr, bend right=50] (\bzLabel{\nodeN}.mid east);
}

\iBzH[1.5]

\bzOn{
	\bzLeft{\hint you cannot draw relatively to a node that has disappeared :)}
}

\iBzH[0.5]

\bzOn{
	\bzLeft{You may use \Verb"calc"-based references, like:}
	\bzCode{\vbBzLabelDrawT}
	\draw (8, \bzH) edge[corarr, bend right=50] ($(\bzLabel{\nodeN}.east)+(4,0)$);
}

\iBzH[1.5]

\bzOn{
	\bzLeft{You may also use the great \Verb"|-" and \Verb"-|" features of \TikZ:}
	\bzCode{\vbBzLabelDrawU}
	\coordinate (base) at ($(\bzLabel{\nodeN}.east)+(0,-0.5)$);
	\draw
		(\bzLabel{\nodeN}.west |- base)
		edge[bzBraceU] node{the one}
		(base -| \bzLabel{\nodeN}.east);
}
\end{bzFrame}


% =======================================================================================

\begin{myverbbox}{\vbBraces}
\path (-4, -11) edge[bzBraceU] node{under brace} (+4, -11);
\path (-4, -6)  edge[bzBraceO] node{over brace}  (+4, -6);
\path (-5, -10) edge[bzBraceL] node{left brace}  (-5, -7);
\path (+5, -10) edge[bzBraceR] node{right brace} (+5, -7);
\end{myverbbox}


\begin{myverbbox}{\vbBzBraces}
\bzCenter{First}
\bzBraceU{1}
\bzNext{\ }

\bzNext{Second}
\bzEvalInt{\secondN}{\bzN}
\bzNext{\ }

\bzNext{Third}
\bzBraceU{3}

\bzBraceU{2}[\secondN]
\end{myverbbox}

\begin{bzFrame}%[show]
\bzCoords

\bzOn{
	\bzCenter[bzEB]{Braces}
}

\bzOn{
	\bzLeft{Use the following commands to draw braces:}
}

\bzOn{
	\bzCode{\vbBraces}
	\path (-4, -6) edge[bzBraceU] node{under brace} (+4, -6);
	\path (-4, -5) edge[bzBraceO] node{over brace}  (+4, -5);
	\path (-5, -7) edge[bzBraceL] node{left brace}  (-5, -4);
	\path (+5, -7) edge[bzBraceR] node{right brace} (+5, -4);
}

\iBzH[6]

\dBzH[0.25]

\bzOn{
	\bzLeft{You can also create braces similarly to \vbBzNext:}
}

\dBzH[0.25]

\bzOn{
	\bzCode{\vbBzBraces}
}

\iBzH[2.5]

\bzOn{
	\bzCenter{First}
	\bzBraceU{1}
	\bzNext{\ }

	\bzNext{Second}
	\bzEvalInt{\secondN}{\bzN}
	\bzNext{\ }

	\bzNext{Third}
	\bzBraceU{3}

	\bzBraceU{2}[\secondN]
}

\end{bzFrame}

% =======================================================================================

\begin{myverbbox}{\vbBlock}\bzBlock{...}{...}\end{myverbbox}
\begin{myverbbox}{\vbBlockOpt}\bzBlock[fill=red]{...}{...}\end{myverbbox}

\begin{myverbbox}{\vbBlockEx}
\bzBlock{0.5}{0.5}
\bzCenter{A simple block!}
\end{myverbbox}

\begin{myverbbox}{\vbBlockTop}
\bzBlock[draw=none, fill=\bzCtext, opacity=0.4]{0.5}{0.5}
\bzBlock[]{0.5}{1.5}
\bzCenter[bzEB]{Beamer-like blocks}
\bzLeft{You can draw blocks like in Beamer.}
\end{myverbbox}

\begin{bzFrame}%[show]
\bzOn{
	\bzBlock[draw=none, fill=\bzCtext, opacity=0.4]{0.5}{0.5}
	\bzBlock[]{0.5}{1.5}
	\bzCenter[bzEB]{Beamer-like blocks}
}

\iBzI[0.5]

\bzOn{
	\bzLeft[]{You can draw blocks like in Beamer.}
}

\zBzI

\iBzH[0.25]

\bzOn{
	\bzLeft{The basic command is \vbBlock{}:}
	\dBzH[0.25]
	\bzCode{\vbBlockEx}
	\iBzH[1.5]
	\bzLeft{which produces a simple block with border, positioned relatively to \vbBzH:}
	\bzBlock{0.5}{0.5}
	\bzCenter{A simple block!}
}

\iBzH[0.25]

\bzOn{
	\bzLeft{The two arguments of \vbBlock{} measure the height of the block:}
	\dBzH[0.25]
}

\bzOn{
	\bzItem{the first one above the current value of \vbBzH{} up to the top,}
	\dBzH[0.25]
}

\bzOn{
	\bzItem{the second one below the current value of \vbBzH{} down to the bottom.}
}

\bzOn{
	\bzLeft{Moreover, the first optional argument is styling, like:}
	\dBzH[0.25]
	\bzCenter{\vbBlockOpt{}}
}

\bzOn{
	\bzLeft{The topmost block of this frame is obtained as:}
	\dBzH[0.25]
	\bzCode{\vbBlockTop}
}
\end{bzFrame}

% =======================================================================================

\begin{myverbbox}{\vbBzLarge}\documentclass[large]{beamerikz}\end{myverbbox}
\begin{myverbbox}{\vbBzNormal}\documentclass{beamerikz}\end{myverbbox}
\begin{myverbbox}{\vbBzMultipleOptions}\documentclass[large, small, basic, large, basic, large, basic]{beamerikz}\end{myverbbox}


\begin{bzFrame}%[show]
\bzOn{
	\bzCenter[bzEB]{Class options --- font size and \Verb"plain"}
}

\bzOn{
	\bzLeft{The class \Verb"BeamerikZ" accepts a number of class options.}
}

\bzOn{
	\bzLeft{The first category are options regarding \hl{font size}:}
}

\bzOn{
	\bzItem{\Verb"small" defines font size of 14pt (smaller than usually),}
}

\bzOn{
	\bzItem{\Verb"basic" is the default option with the font of 17pt used,}
}

\bzOn{
	\bzItem{\Verb"large" uses font of 20pt.}
}

\bzOn{
	\bzLeft{Specify these options when invoking document class, e.g.:}
	\bzCode{\vbBzLarge}
}

\iBzH

\bzOn{
	\bzLeft{This document is typeset using the default option of 17pt, i.e. with:}
	\bzCode{\vbBzNormal}
}

\iBzH

\bzOn{
	\bzBox{All the class options are processed sequentially, so the following code gives the default~17pt font size:}
	\iBzH[0.8]
	\bzCode{\vbBzMultipleOptions}
}

\iBzH

\bzOn{
	\bzBox{If you really hate ``Powered by \beamerikz'' on the first plain frame, use the class option \Verb"plain".}
}
\end{bzFrame}

% =======================================================================================

\begin{myverbbox}{\vbBzWide}\documentclass[wide]{beamerikz}\end{myverbbox}

\begin{bzFrame}%[show]
\bzOn{
	\bzCenter[bzEB]{Class options --- 16:9}
}

\bzOn{
	\bzLeft{Package option \Verb"wide" sets the aspect ratio of the presentation to 16:9}
	\bzCode{\vbBzWide}
}

\iBzH

\bzOn{
	\bzLeft{It {\bf does not} change the size of the text!}
	\bzText{In particular, you can still use options \Verb"small", \Verb"basic", and \Verb"large"}
}

\bzOn{
	\bzLeft{Also, the vertical space is unchanged: (-1.0 -- 14.5)}
}

\bzOn{
	\bzLeft{However, the horizontal space is then: (-13.0 -- +13.0) (plus margins)}
}

\bzOn{
	\bzLeft{This means that the default value of \Verb"bzI" becomes -13}
}
\end{bzFrame}

% =======================================================================================

\begin{myverbbox}{\vbFrameShow}
\begin{bzFrame}[show]
...
\end{bzFrame}
\end{myverbbox}

\begin{myverbbox}{\vbFrameHide}
\begin{bzFrame}%[show]  % to make this frame [show] again, remove the first % in this line
...
\end{bzFrame}
\end{myverbbox}


\begin{myverbbox}{\vbFrameHidePlain}
\begin{bzPlainFrame}%[show]  % to make this frame [show] again, just remove %
...
\end{bzPlainFrame}
\end{myverbbox}


\begin{bzFrame}%[show]
\bzOn{
	\bzCenter[bzEB]{Class options --- \Verb"[show]" frames}
}

\bzOn{
	\bzLeft{Compilation of complex presentations may take up to a couple of minutes!}
}

\iBzH[0.3]

\bzOn{
	\bzBox{To speed-up the development process, \Verb"BeamerikZ" provides a mechanism, that allows do disable compilation of certain frames to focus on the currently developed ones.}
	\iBzH[2]
}

\bzOn{
	\bzLeft{Frames of interest should be marked with the option \Verb"[show]":}
	\bzCode{\vbFrameShow}
	\iBzH[2.0]
}

\bzOn{
	\bzLeft{While the rest should have this option commented out:}
	\bzCode{\vbFrameHide}
	\iBzH[2]
}

\bzOn{
	\bzLeft{The same option applies to the ``plain'' frames:}
	\bzCode{\vbFrameHidePlain}
	\iBzH[1.8]
}

\bzOn{
	\bzLeft{\hint to make use of these options, read on!}
}
\end{bzFrame}

% =======================================================================================

\begin{myverbbox}{\vbFrameShow}
\begin{bzFrame}[show]
...
\end{bzFrame}
\end{myverbbox}

\begin{myverbbox}{\vbFrameHide}
\begin{bzFrame}%[show]  % to make this frame [show] again, remove the first % in this line
...
\end{bzFrame}
\end{myverbbox}


\begin{myverbbox}{\vbFrameHidePlain}
\begin{bzPlainFrame}%[show]  % to make this frame [show] again, just remove %
...
\end{bzPlainFrame}
\end{myverbbox}


\begin{bzFrame}%[show]
\bzOn{
	\bzCenter[bzEB]{Class options --- compilation modes}
}

\bzOn{
	\bzLeft{Similarly as \Verb"normal" and \Verb"large", \Verb"BeamerikZ" reacts to the following options:}
}

\bzOn{
	\bzItem{\Verb"final" --- the default option, when all the frames are normally compiled}
}

\bzOn{
	\bzItem{\Verb"ready" --- similar to \Verb"final", but add compile info (see later)}
}

\bzOn{
	\bzItem{\Verb"draft" --- compiles \Verb"[show]" frames, others are made one-shot}
}

\bzOn{
	\bzItem{\Verb"short" --- compiles all the frames in one-shot, \Verb"[show]" is ignored}
}

\bzOn{
	\bzItem{\Verb"brief" --- compiles only \Verb"[show]" frames, others are blank}
}

\iBzH[0.5]

\bzOn{
	\bzLeft{Thus, in terms of compilation time we get ({\small assuming few \Verb"[show]" frames}):}
	\bzCenter{\Verb"brief" $<$ \Verb"short" $\leq$ \Verb"draft" $<$ \Verb"final" $\leq$ \Verb"ready"}
}
\end{bzFrame}

% =======================================================================================

\begin{bzFrame}%[show]
\bzOn{
	\bzCenter[bzEB]{Class options --- \Verb"draft" and \Verb"brief"}
}

\bzOn{
	\bzLeft{The modes \Verb"draft" and \Verb"brief" are meant to help with development of slides.}	
}

\bzOn{
	\bzLeft{For that purpose, they feature:}
}

\bzOn{
	\bzList{A grid of coordinates on each \Verb"[show]" frame to help arrange the nodes.}
}

\bzOn{
	\bzList{A slide counter in the bottom left of each \Verb"[show]" frame.}
}

\bzOn{
	\bzText{\hint use this counter to synchronize \vbBzIn{} arguments!}
}

\bzOn{
	\bzLeft{Moreover, \Verb"draft" forces Beamer to use \Verb"draft" to speed-up the compilation process.}
}
\end{bzFrame}

% =======================================================================================


\begin{myverbbox}{\vbPlainFirst}
\begin{bzPlainFrame}
...
\end{bzPlainFrame}
\end{myverbbox}


\begin{bzFrame}%[show]
\bzOn{
	\bzCenter[bzEB]{Class options --- \Verb"ready"}
}

\bzOn{
	\bzLeft{The last thing to be explained is the \Verb"ready" class option.}
}

\bzOn{
	\bzLeft{It works exactly as \Verb"final" except for the first \Verb"bzPlainFrame":}
}

\bzOn{
	\bzCode{\vbPlainFirst}
	\iBzH[2]
}

\bzOn{
	\bzBox{Where it adds an additional \vbBzOn{} with the time of the current recompilation, written on the bottom of the slide.}
	\iBzH
}

\bzOn{
	\bzLeft{It serves two purposes:}
}

\bzOn{
	\bzItem{Makes it easier for you to make sure which version you are to present.}
}

\bzOn{
	\bzItem{Allows you to test your slide-switcher without unravelling the next slide :)}
}
\end{bzFrame}

% =======================================================================================

\begin{bzFrame}%[show]
\bzOn{
	\bzCenter[bzEB]{Credits}
}

\bzOn{
	\bzLeft{The author would like to thank:}
}

\bzOn{
	\bzItem{Till Tantau for his amazing work on Beamer and PGF/\TikZ!}
}

\bzOn{
	\bzItem{Szczepan Hummel for suggestions and a prototype of \vbBzNext,}
}

\bzOn{
	\bzItem{Filip Mazowiecki and Micha{\l} Pilipczuk for never-ending enthusiasm,}
}

\bzOn{
	\bzItem{Kamila {\L}yczek for support,}
}

\bzOn{
	\bzItem{Abhishek Aich and Bartosz Bednarczyk for feature requests.}
}
\end{bzFrame}

% =======================================================================================

\end{document}
